\documentclass{article}
\usepackage[utf8]{inputenc}
\usepackage{babel}
\usepackage{polski}
\usepackage{float}
\usepackage{graphics}
\title{Wykład 2}
\maketitle
\begin{document}

	\section{Tworzenie modelu obiektu wielowymiarowego w przestrzeni stanu cz. 2}
	\subsection{Ustalanie wymiaru wektora stanu na podstawie równań wejścia-wyjścia}

		\begin{equation}
			y^{(n)}(t) + a_{n-1} y^{(n-1)}(t) + ... +   a_{0} y(t) = b_{(m)} u(t) + ... + b_{0} u(t)
		\end{equation}

		Ponieważ jest to równanie liniowe n-tego rzędu, to jego rozwiązanie wymaga znajomości
		n warunków początkowych. Liczba warunków początkowych wyznacza wymiar wektora stanu
		i przestrzeni stanu X, czyli dim(X) = n

		\begin{equation}
			x^{T} = [ x_{1} x_{2} ... x_{n} ]
		\end{equation}

		Od tej pory równania macierzowe będą odpowiednio podpisane aby się nie myliły

	\subsection{Definicja zmiennych stanu jako zmiennych fazowych}

		Przyjmijmy następującą definicje współrzędnych wektora zmiennych stanu x
		dla układu o __jednym wejściu__

		$ x_{1} = y $
		$ x_{2} = y'$
		...
		$ x_{n} = y^{(n-1)}$

		Z powyższej definicji wynika że

		$ x_{i+1} = \dot{x_{i}} $

		Dla i = <1;n-1>

		Dla wielu wyjść powyższe równania możemy zapisać analogicznie

		Dla wyjścia 1

		$ y_{1} = x_{1} $
		$ \dot{y_{1}} = x_{2} $
		...
		$ y_{1}^{(n_{1}-1)} = x_{n_{1}} $

		$ y_{2} = x_{n_{1}+1} $
		$ \dot{y_{2}} = x_{n_{1}+2} $
		...
		$ y_{2}^{(n_{2}-1)} = x_{n_{1}+n_{2}} $

		Aż do n_{p}

		$ y_{p} = x_{n_{1}+...+n_{p-1} + 1} $
		$ \dot{y_{p}} = x_{n_{1}+...+n_{p-1} + 2} $
		...
		$ y_{n}^{(n_{p}-1)} = x_{n_{1}+...+n_{p}} $

		(przyp. tł. Zrozumienie tych równań nie jest krytyczne dla dalszej części,
		jednak na wykładzie są one napisane z błędem oraz dziwnym sposobem, gdzie
		indeks dolny jest zastępowany kolorem zielonym. )

		Rezultatem takiej definicji wektora zmiennych stanu jest to że macierz modelu A
		ma postać macierzy Frobeniusa, czyli

		\begin{equation}
			A=\begin{pmatrix}
			  0    &   1    &   0    &   0    & \cdots &   0      &   0    \\
			  0    &   0    &   1    &   0    & \cdots &   0      &   0    \\
			  0    &   0    &   0    &   1    & \cdots &   0      &   0    \\
			  0    &   0    &   0    &   0    & \cdots &   0      &   0    \\
			\vdots & \vdots & \vdots & \vdots & \ddots & \vdots   & \vdots \\
			  0    &   0    &   0    &   0    & \cdots &   0      &   1    \\
			 -a_{1}&  -a_{2}&  -a_{3}&  -a_{4}& \cdots & -a_{m-1} & -a_{m}
			\end{pmatrix}
		\end{equation}

		Macierz wejść ma podobną strukturę jednak nie jest to macierz Frobeniusa
		\begin{equation}
			B=\begin{pmatrix}
			  b_{1} &   0    &   0    &   0    & \cdots &   0      &   0    \\
			  0     &  b_{2} &   0    &   0    & \cdots &   0      &   0    \\
			  0     &   0    & b_{3}  &   0    & \cdots &   0      &   0    \\
			  0     &   0    &   0    &  b_{4} & \cdots &   0      &   0    \\
			\vdots  & \vdots & \vdots & \vdots & \ddots & \vdots   & \vdots \\
			  0     &   0    &   0    &   0    & \cdots &  b_{m-1} &   0    \\
			  0     &   0    &   0    &   0    & \cdots &   0      &   b_{m}
			\end{pmatrix}
		\end{equation}

		Natomiast macierz wyjść jest wierszowa
		\begin{equation}
			C=\begin{pmatrix}
			  0    &   0    &   0    &   0    & \cdots &   0      &   1
			\end{pmatrix}
		\end{equation}
		Układ równań stanu i układ równań wyjścia wielowymiarowego obiektu sterowania o
		liniowych właściwościach statycznych i dynamicznych można zapisać w postaci
		równań macierzowych

		$ \dot{x} = Ax + Bu $
		$ y       = Cx + Du $

		Te zawiłe równania doprowadziły nas do równań stanu. Przypomnijmy sobie ich nazwy

		- A : macierz układu
		- B : macierz sterowań
		- C : macierz wyjść
		- D : macierz transmisyjna

		Powrót do transmitancji z równań stanu jest trywialny
		charakteryzuje go jeden wzór

		\begin{equation}
			G(s) = C[s \cdot I - A]^{-1}B + D
		\end{equation}

		Jednak przejście z transmitancji na równania stanu nie jest już tak jednoznaczne.
		Wychodząc od transmitancji można zastosować jedną ze znanych z podstaw
		automatyki metod generowania tzw. modeli analogowych
		np. bezpośrednią, albo iteracyjną, albo równoległą i na ich podstawie
		zdefiniować zmienne stanu, a następnie określić na podstawie schematu poszczególne
		elementy macierzy A,B,C,D.
		Jednak będą to różne macierze które będą równoważne

		Dla układów typu SISO wymienione metody są stosunkowo proste nawet w przypadku
		obiektów wysokiego rzędu, które mają złożone, lecz ciągle liniową dynamikę.
		W przypadku układów typu MIMO niezbędny jest dodatkowy proces unifikacji
		zmiennych stanu, które mogą się powtarzająć się w grafach przepływu sygnałów
		poszczególnych kanałów wejścia-wyjścia. Unifikacja jest niezbędna dla usunięcia
		redundancji (nadmiaru) w opisie obiektu dynamicznego w przestrzeni stanu, gdyż
		zgodnie z definicją, wektor stanu opisujący obiekt dynamiczny musi
		mieć rozmiar minimalny.

	\section{Definicja zmiennych fazowych – przypadek obecności zer w transmitancjach kanałów}

	Przypomnijmy sobie równanie

	\begin{equation}
		 M_{ij}(s) \cdot Y_{j}(s) = L_{ij}(s) \cdot U_{i}(s) 
	\end{equation}
	Jak widać mamy pochodne po prawej stronie równania, co oznacza że dla pewnych
	przebiegów sygnał będzie zerować stan.
	Po automatycznemu nazywamy taką sytuacje że mamy "zera" w transmitancji.

	\subsection{Skąd się te zera wzięły skoro nie było ich w fizycznym obiekcie?}

	Model wielokanałowy układu MIMO zakłada, że przepływ sygnałów wejściowych do wyjść
	odbywa się "do przodu". Rozważany sygnał wejściowy przepływa przez
	kolejne połączone szeregowo człony dynamiczne i jego wartość chwilowa jest
	modyfikowana.

	Człon dynamiczny często charakteryzuje się występowaniem lokalnego
	sprzężenia zwrotnego. Niech jego transmitancja w stanie otwartym będzie równa:
	$G(s) = L(s)/M(s)$, a bieguny są różne od zer.

	W układzie zamkniętym pętlą prostego sprzężenia zwrotnego

	\begin{equation}
		G(s) = \frac{L(s)}{M(s) + L(s)}
	\end{equation}

	Położenie biegunów układu zamkniętego ulega modyfikacji, bo równanie charakterystyczne
	ma obecnie postać:
	\begin{equation}
		M(s) + L(s) = 0
	\end{equation}

	Współczynniki wielomianu równania charakterystycznego układu zamkniętego zostały
	zmodyfikowane od indeksu 0 aż do m :

	Dla i = 0, 1, ..., m

	\begin{equation}
		\dot{a_{i}} = a_{i} + b_{i}
	\end{equation}
	Wraz z naszą definicją macierzy A oznacza to że mamy oddziaływania wskrośne
	pomiędzy kanałami oraz że mamy "wirtualne wymuszenie" zewnętrzne w postaci

	\begin{equation}
		\dot{u_{i}}(t) = b_{m}u_{i}(t)^{(m)} + b_{m-1}u_{i}(t)^{(m-1)} + ... + b_{0} u_{i}(t) $
	\end{equation}

	Rozumienie takich oddziałowywań jako "wymuszeń wirtualnych" nakłada istotne ograniczenia
	na sposób sterowania bo musimy przy naszym planowaniu uwzględniać pochodne sygnału
	sterującego.

	Warto zauważyć że pomimo tych ograniczeń nie zmieniamy właściwości dynamicznych
	rozważanego kanału jak stabilność bo równanie po lewej stronie nie zostało zmienione
	Wpływa to jednak na takie cechy jak sterowalność oraz obserwowalność.

	Obecność zer również ogranicza nam możliwość skorzystania ze sterowania
	cyfrowego/schodkowego. Przez to że pochodne będą nam wzrastać do nieskończoności
	przy takim schodku.

	Widać, ze nawet, gdy dany kanał nie ma zer, to obecność np. regulatora PID jako lokalnego
	stabilizatora urządzenia wprowadzi zera, w dodatku przy nastrajanych nastawach
	ze względu na potrzebę stabilizacji obiektu w danym punkcie zera te zmieniają
	położenie na płaszczyźnie zespolonej i mogą koincydować (pokrywać się) z
	biegunami danego kanału, niekoniecznie związanymi z danym urządzeniem.

	### Definicja zmiennych stanu gdy w transmitancji obiektu występują zera

	Dla uproszczenia poniższe rozważania będą rozpisane dla jednego kanału
	wejścia-wyjścia jednak są one prawdziwe dla wszystkich kanałów.

	Należy zmienić definicję zmiennych stanu nieznacznie z formy

	$ x_{1} = y $
	$ x_{2} = y'$
	...
	$ x_{n} = y^{n-1}$

	Na

	$ x_{1} = y - \beta_{0}u $
	$ x_{2} = y' - \beta_{1}u$
	...
	$ x_{n} = y^{(n-1)} - \beta_{n-1}u$

	W wyniku powyższej definicji zmiennych fazowych równanie wejścia-wyjścia
	zostaje przekształcone do postaci
	$ y^{(n)}(t) = \dot{x_{n}}(t) = -a_{1}x_{1}(t) - a_{2}x_{2}(t) + ... + a_{n}x_{n}(t) + \beta_{m}u(t) $

	Wówczas współczynniki \beta_{k} wynoszą

	$ \beta_{1} = b_{n} $
	$ \beta_{2} = b_{n-1} - a_{n-1}b_{0}$
	...
	$ \beta_{n} = b_{1} - a_{n-1}b_{n-1} - ... - a_{1}b_{1}$

	Taki układ ma wtedy postać taką że

	- Macierz A ma postać Frobeniusa
	- B = [\beta_{2} ... \beta_{n}]^{T}
	- C = [0 0 0 0 1]
	- D = [\beta_{1}
\end{document}
