\author{Kasiński}
\title{Wykład 5}
\documentclass{article}
\usepackage[utf8]{inputenc}
\usepackage{babel}
\usepackage{polski}
\usepackage{float}
\usepackage{graphicx}
\usepackage{amsmath}
\begin{document}
	\maketitle
	\section{Linearyzacja modelu w otoczeniu punktu stacjonarnego}
	\subsection{Wstęp}
		Linearyzację modelu ogólnego przeprowadzimy dla otoczenia wybranego punktu
		stacjonarnego $x_s^i$. Dla przejrzystości i bez utraty ogólności rozważań, w następujących
		dalej wzorach, pomińmy indeks górny $i$, wskazujący konkretny punkt stacjonarny oraz
		indeks dolny $s$ informujący, że chodzi o punkt stacjonarny jako punkt referencyjny .
		Zakładamy, że w tym punkcie u(0) = 0. Używany w dalszej części symbol normy oznacza
		normę Euklidesa oznaczonej wzorem
		\begin{equation}
			||P|| = \sqrt{P.x^2 + P.y^2}
		\end{equation}

		Otoczenie tego punktu możemy sobie wyobrazić jako n-wymiarową hiperkulę o środku $x$
		oraz o małym promieniu $\varepsilon$ gdzie $||\varepsilon|| << 1$.
		To mniej więcej oznacza, że otoczenie punktu jest nie dalej niż $\varepsilon$ od punktu $x$.
		
		Dla jasności wektor prędkości zmian stanu będziemy nazywać momentum.
		Do linearyzacji modelu w otoczeniu punktu stacjonarnego posłużymy się szeregiem Taylora.
		\begin{equation}
			\dot{z}(t) = f(z) = \sum_{n=0}^{\infty}\frac{f^{(n)}}{n!}(z-x)^n
		\end{equation}
		gdzie: $z$ oznacza dowolny wektor wewnątrz hiperkuli oznaczonej przez $B_n(x, \varepsilon)$, a 
		$(z-x)$ oznacza odchylenie od punktu stacjonarnego takie, że $||z-x||<\varepsilon$.
		$f^{(n)}$ oznacza dla różnych wartości $n$ różne rzeczy.
		\begin{itemize}
			\item dla $n=0$ $f^{(n)}$ oznacza wartość momentum w punkcie stacjonarnym.
				Z definicji ta wartość jest wektorem zerowym.
			\item dla $n=1$ $f^{(n)}$ oznacza Jakobian $J(x)$ wektorowego momentum, który dla 
				punktu stacjonarnego wynosi $\frac{\delta f}{\delta z^T}$
			\item dla $n=2$ $f^{(n)}$ oznacza formę kwadratową $(z-x)^TH(x)(z-x)$
				gdzie $H(x)$ oznacza Hesjan $\frac{\delta^2f}{\delta z^t \delta z}$
			\item dla $n>2$ są to formy tensorowe wyższych rzędów
		\end{itemize}
		Z założonej analityczności
		$f$ wynika, że normy tensorów, w tym Jacobianu i Hessjanu, 
		(wartości wyznaczników) są skończone. Oznaczmy je odpowiednio przez $J$ i $H$.
	\subsection{Obcięcie rozwinięcia i majoryzacja}
		Ograniczmy rozwinięcie w szereg Taylora do trzech pierwszych wyrazów i przejdźmy do
		obliczenia normy po obu stronach równania:
		\begin{equation}
			||\dot{z}(t)||=||f(z)|| = ||f(x) + J(x)(z-x) + (z-x)^TH(x)(z-x)||
		\end{equation}

		\begin{equation}
			||f(x) + J(x)(z-x) + (z-x)^TH(x)(z-x)|| \leq ||f(x)|| + J||(z-x)|| + ||(z-x)^T||H||(z-x)||
		\end{equation}
		co oznacza 
		\begin{equation}
			||\dot{z}(t)|| \leq J \varepsilon + H \varepsilon 
		\end{equation}
		$J$ oraz $H$ są skończone oraz prawdziwe są nierówności
		\begin{equation}
			... \leq \varepsilon^3 \leq \varepsilon^2 \leq \varepsilon \leq 1
		\end{equation}
		Natomiast $f(x)$ w punkcie stacjonarnym jest wektorem zerowym. Na tej podstawie możemy
		wywnioskować, że dla oszacowania wartośći wektora prędkośći zmian stanu, nazywanym również wektorem momentum, możemy ograniczyć się do zlinearyzowanego modelu
		\begin{equation}
			\dot{z}(t)=Az(t)
		\end{equation}
		gdzie $A=J$.
		$A$ jest macierzą stanu o stałych elementach. Dla modelu zlinearyzowanego jest ona równa
		Jacobianu funkcji momentum obliczonemu w punkcie $x$.
		$z(t)$ jest wektorem stanu układu zlinearyzowanego, pozostającym hiperkuli.Dla modelu zlinearyzowanego jest ona równa
		Jacobianu funkcji momentum obliczonemu w punkcie $x$.
		$z(t)$ jest wektorem stanu układu zlinearyzowanego, pozostającym hiperkuli.
		Dla ułatwienia rozważań można przesunąć równolegle układ współrzędnych stanu w
		taki sposób, by początek przesuniętego układu przypadał w wybranym punkcie
		stacjonarnym $x$, Wówczas odchylenia od punktu stacjonarnego $(z – x)$ można uznać za
		współrzędne stanu wyrażone w nowym układzie współrzędnych, co upraszcza powyższe
		wzory rozwinięcia w szereg.
		
	\subsection{Rozszerzenie na obiekt sterowalny - lokalny model zlinearyzowany}
		Analogiczne rozumowanie można zastosować do układów nieautonomicznych.
		\begin{equation}
			\dot{x}(t) = f(x(t), u(t))
		\end{equation}
		Zakładając że sterowania nie wyprowadzają punktu z hiperkuli $B_m$, a wyznaczniki 
		$J$ oraz $H$ są skończone, a $x$ oznacza wektor stanu w nowym przeniesionym układzie współrzędnych.
		Możemy zapisać zlinearyzowany model dynamiki obiektu sterowania.
		\begin{equation}
			\dot{x} = Ax(t) + Bu(t)
		\end{equation}
		Ten zabieg możemy również powtórzyć w przestrzeni wyjść $Y$ gdzie
		\begin{equation}
			y = h(x, 0)
		\end{equation}
		Wówczas zlinearyzowane lokalnie w $y$ równanie wyjścia w nowym układzie współrzędnych
		w którym to układzie $h(x, 0) = 0$
		\begin{equation}
			y(t) = Cx(t) + Du(t)
		\end{equation}
		W tym wypadku macierz wyjścia oraz macierz przeniesienia wyglądają następująco
		\begin{align*}
			C = \frac{\delta h}{\delta x^T} \\
			D = \frac{\delta h}{\delta u^T} 
		\end{align*}
		Macierze te są oczywiście obliczone w punkcie 
		\begin{equation}
			\begin{pmatrix}
				x & 0
			\end{pmatrix}
		\end{equation}
		Macierze $A,B,C,D$ nazywamy
		liniową realizacją modelu wielowymiarowego obiektu
		dynamicznego w wybranym układzie współrzędnych stanu, przyporządkowaną
		konkretnemu, wybranemu punktowi stacjonarnemu ogólnego nieliniowego modelu
		dynamiki obiektu sterowania.
	\subsection{Niejednoznaczność modelu lokalnego}
		Ponieważ wybór zmiennych stanu dla danego obiektu nie jest jednoznaczny, to możliwe
		jest uzyskanie wielu różniących się od siebie liniowych realizacji modelu w tym samym
		punkcie stacjonarnym, gdyż:
		\begin{itemize}
			\item Zmienne stanu w wektorze mogą być umieszczone w różnej kolejności
			\item Zmienne stanu mogą być wyrażone w różnych jednostkach
			\item Zmienne stanu mogą pochodzić z bezpośredniego opisu elementów fizycznych
				połączonych w schemat blokowy
			\item Zmienne stanu mogą pochodzić z analizy równań wejścia-wyjścia i być
				zdefiniowane jako zmienne fazowe
			\item Zmienne stanu mogą reprezentować warunki początkowe arbitralnie
				uszeregowanych równań wejścia wyjścia
		\end{itemize}
	\subsection{Związek z macierzową transmitancją}
		Każdą poprawnie obliczoną realizację liniową można łatwo sprowadzić do
		macierzowej transmitancji operatorowej.

		W przypadku obiektu typu SISO, taką transmitancję łatwo obliczyć w stosując
		obustronne przekształcenie Laplace'a do pojedynczego równania wejścia-wyjścia
		w dziedzinie czasu. Musimy założyć że warunki początkowe są równe zeru.

		W przypadku obiektu typu MIMO, ten spobób nie ma zastosowania przez nieokreśloność
		dzielenia przez wektor transformat sygnału wejściowego.
		Transmitancję $\underline{G(s)}$ definiujemy w następujący sposób:
		\begin{equation}
			Y(s) = \underline{G(s)}U(s)
		\end{equation}
		Gdzie $U(s)$ oraz $Y(s)$ są wektorami transformat Laplace'a wektora sygnałów
		wejściowych i wektora sygnałów wyjściowych. Macierz $\underline{G(s)}$
		ma rozmiary zależne od wejść oraz wyjść sygnałów, a jej elementy są ilorazami
		wielomianów zmiennej zespolonej $s$.

		Stosując obustronne przekształcenie Laplace'a do równań stanu i równań wyjścia
		modelu zlinearyzowanego otrzymujemy
		\begin{align*}
			sX(s)-x(0) = AX(s) + BU(s) \\ 
			Y(s) = CX(s) + DU(s)
		\end{align*}
		Z założeń stransmitancji przyjmujemy zerowe warunki początkowe, stąd mamy
		\begin{equation}
			sX(s) - A X(s) = BU(s)
		\end{equation}
		Wyciągając $X(s)$ przed nawias otrzymujemy
		\begin{equation}
			[sI-A]X(s) = BU(s)
		\end{equation}
		Po dalszych przeniesieniach możemy orzymać $x(t)$
		\begin{equation}
			x(t) = \mathcal{L}^{-1}\{ [sI-A]^{-1}BU(s)\}
		\end{equation}
		$x(t)$ jest częścią wymuszoną, a dokładniej jest wymuszoną przez u(t).

		Warto tutaj zauważyć, że nieosobliwy model liniowy ma w przestrzeni czasu 
		tylko jeden punkt stacjonarny mędący początkiem układu, czyli $x=0$.
		Jest to logiczne jeśli przypomnimy sobie, że $Ax=0$ ma jedno roziwiązanie
		gdy $det(A) \neq 0$ czyli $x=0$

		Gdy macierz $A$ jest osobliwa możemy powiedzieć że charakteryzuje się defektem rzędu.
		Wtedy rozwiązaniem równania $Ax=0$ jest dowolny niezerowy wektor należący do 
		jądra macierzy A.
		
		Wracając do równania
		\begin{equation}
			Y(s) = \{C[sI-A]^{-1}B+D\}U(s)
		\end{equation}
		Zgodnie z definicją otrzymujemy $\underline{G(s)}$ 
		\begin{equation}
			\underline{G(s)} = C[sI-A]^{-1}B+D
		\end{equation}
		Macierz odpowiedzi impulsowych poszczególnych kanałów łączących wejścia z
		wyjściami jest równa:
		\begin{equation}
			g_{ij}(t) = \mathcal{L}^{-1}\{ \underline{G(s)} \}
		\end{equation}
		gdzie $i$ jest indeksem wejścia a $j$ indeksem wyjścia
	\subsection{Rozwiązywanie równań stanu}
		W poniższych równaniach będzie wyprowadzony spobób aby otrzymać 
		stan obiektu w dziedzinie czasu posiadając model liniowy obiektu.
		\begin{align*}
			\dot{x}(t) = Ax(t)+Bu(t) \\
			sX(s) - x(0) = AX(s) + BU(s) \\
			X(s) = [sI-A]^{-1}x(0) + [sI-A]^{-1}BU(s) \\
			x(t) = \mathcal{L}^{-1}\{ [sI-A]^{-1}x(0) \} + \mathcal{L}^{-1}\{ [sI-A]^{-1}BU(s) \} 
		\end{align*}
		Pierwszy składnik ma oryginał transformaty równy $e^{At}x(0)$ i przedstawia przebieg
		trajektorii stanu układu autonomicznego, wynikacjyący z niezerowego stanu początkowego.
		Druga transformata odwrotna wymaga obliczenia oryginału iloczynu dwóch macierzowych funkcji zespolonych: $[sI-A]^{-1}$ oraz $BU(s)$. Iloczynowi odpowiada oryginał w postaci całki
		splotowej
		\begin{align*}
			\mathcal{L}^{-1}\{ [sI-A]^{-1}BU(s) \} \\
			e^{At}\ast Bu(t) \\
			\int^{\infty}_0e^{A(t-\tau)}Bu(\tau)d\tau
		\end{align*}
		Powyższa część równania stanu jest tą częścią która wymusza sygnały wejściowe
		na obiekcie. Dopiero znając przebieg tych sygnałów jesteśmy w stanie 
		wyznaczyć 
		\begin{equation}
			x(t) = e^{At}x(0)+\int^t_0e^{A(t-\tau)}Bu(\tau)d\tau
		\end{equation}
		gdzie $t$ oznacza chwilę bierzącą, a $\tau$ oznacza chwilę z przeszłości.
		Funkcja $e^{A(t-\tau)}$ jest nazywana funkcją "zapominania" efektów sterowania
		z przeszłości.
		Funkcję $e^{At}$ nazywamy macierzą tranzycji stanu liniowego stacjonarnego układu o
		macierzy stanu $A$.
	\subsection{Dygresja o macierzowej funkcji wykładniczej}
		Skalarna funkcja wykładnicza $e^x$ jest definiowana jako granica szeregu
		nieskończonego
		\begin{equation}
			e^x = \sum^{\infty}_{n=0}\frac{1}{n!}x^n
		\end{equation}
		Wnioskując z tego możemy otrzymać że funkcja wykładnicza $e^{At}$ będzie wyglądała następująco
		\begin{equation}
			e^{At} = \sum^{n-1}_{k=0}\frac{1}{k!}(At)^k = I+\frac{At}{1!} ...
		\end{equation}
		Na mocy twierdzenia Cayleya-Hamiltona rozwinięcie to jest skończone.
		Warto zauważyć tutaj że n oznacza rząd macierzy A.

		Dla czasów niewiele późniejsych niż chwila początkowa, dobrym przybliżeniem
		macierzowej funkcji wykładniczej jest jej aproksymacja pierwszego rzędu
		\begin{equation}
			I+A\Delta t
		\end{equation}
\end{document}
