\author{Kasiński}
\title{Wykład 10}
\documentclass{article}
\usepackage[margin=1.4cm]{geometry}
\usepackage[utf8]{inputenc}
\usepackage{babel}
\usepackage{polski}
\usepackage{float}
\usepackage{graphicx}
\usepackage{amsmath}
\usepackage{hyperref}
% More defined colors
\usepackage[dvipsnames]{xcolor}
% Required package
\usepackage{tikz}
\usetikzlibrary{positioning}
\begin{document}
	\maketitle
	\section{Synteza obserwatora dla rozszerzonego modelu obiektu}
		Do konstrukcji algorytmu obserwatora wykorzystuje się model dyskretny (liniowy)
		obiektu sterowania:
		\begin{align*}
			x_{k+1} = Ax_k + Bu_k \\ 
			y_k = C x_k + Du_k
		\end{align*}
		Obserwator rekonstruuje wartość wektora stanu $x_{k-N}$
		na podstawie danych, w postaci zapamiętanych próbek $y_i$ oraz $u_i$,
		\begin{equation}
			i=k-N, k-N+1, \ldots , k
		\end{equation}
		Równanie różnicowe obserwatora pełnego ma postać:
		\begin{align*}
			z_{k+1} = [A-LC] z_k - Ly_k + Bu_k \\ 
			x^E_k = z_k
		\end{align*}
		gdzie wskaźnik górny $^E$ wskazuje, że jest to estymata wektora stanu.
		Zdefiniujmy wektor residuum układu w następujący sposób:
		\begin{equation}
			r_k = y_k - y^E_k
		\end{equation}
		Gdzie 
		\begin{equation}
			y^E_k = Cx^E_k
		\end{equation}
		Jest to różnica pomiędzy rzeczywistą wartością na wyjściu obiektu w dyskretnej chwili $k$, a
		wartością na wyjściu modelu, obliczona na podstawie oszacowanego stanu w chwili $k$.
		Można wykorzystać obserwator do wykrywania uszkodzeń w obiekcie, objawiających się
		przez wystąpienie wektora sygnałów awaryjnych $f_k$ (zdarzeń) w chwili $k$. Skutki awarii
		propagują się w układzie dynamicznym:
		\begin{equation}
			x_{k+1} = F f_k
		\end{equation}
		Powyższe równanie to model dynamiki awarii. Jeżeli awaria $f_k$ powoduje skutki strukturalne, to
		$F = F_k$ (model niestacjonarny).
		
		Cały czas rozważając model dyskretny możemy zdefiniować
		macierz $F = [ f_{ij} ]$, która jest macierzą o elementach $0$, albo $1$. $f_{ij} = 1$ oznacza, że awaria $j$-tego
		typu ma wpływ na $i$-tą zmienną stanu $i$ skutkuje skokową zmianą wartości prędkości
		zmian $i$-tej zmiennej stanu równą $f_{ij}[k]$ przed wystąpieniem awarii
		$ f_{ij}[0]= f_{ij}[1]= \ldots f_{ij}[k]= 0 $.
		$f_j[k]$ jest ilościową miarą skutku awarii $j$-tego typu w chwili $k$-tej. Na przykład jeżeli
		awaria polega na całkowitym uszkodzeniu sensora mierzącego wartość $i$-tej zmiennej
		stanu, to wyniki pomiaru $x_i[k] = x_i[k] = \ldots x_i[N] = 0$ (takie wartości „trafią” do sytemu
		sterowania, gdy tymczasem rzeczywiste ich wartości nadal ewoluują i wynikają z równań
		dynamiki obiektu. $f_{ij} = 0$ oznacza, że awaria $j$-tego typu nie wpływa na prędkość zmian $i$-
		tej zmiennej stanu.

		Jeżeli obiekt sterowania jest zaburzany przez sygnały zakłócające, a przynajmniej część
		z nich jest mierzalna ($d_k$ - wektor mierzalnych zakłóceń w chwili $k$), to mogą być one
		wykorzystane w procesie obserwacji na równych prawach z sygnałami sterującymi,
		ponieważ również stanowią pobudzenie dynamiczne obiektu, objawiające się skutkami
		na wyjściu.
		\begin{equation}
			x_{k+1} = Ed_k
		\end{equation}
		powyższy model jest modelem propagacji zakłóceń do zmiennych stanu obiektu. Macierz $E$ jest
		macierzą o elementach $0$, albo $1$. $d_{ij} = 1$ oznacza, że $j$-te zaburzenie ma wpływ na $i$-tą
		zmienną stanu i skutkuje skokową zmianą wartości prędkości zmian $i$-tej zmiennej stanu
		równą $d_j[k]$ (przed wystąpieniem zaburzenia $d_j[0]= d_j[1]= \ldots d_j[k-1]= 0$).
		$d_j[k]$ jest amplitudą $j$-tego zaburzenia w chwili $k$-tej. Po wystąpieniu, zaburzenie może
		ewoluować w czasie, czego miarą są wyniki jego pomiaru $d_{ij}[k], dij[k+1], dij[k+2], \ldots $ 

		Rozszerzony model dynamiki obiektu, uwzgledniający ewentualne awarie oraz
		wykorzystujący pomiary zakłóceń ma postać:
		\begin{figure}[H]
			\begin{align*}
				x_{k+1} = Ax_k + Bu_k + Ed_k + Ff_k \\ 
				y_k = C x_k + Du_k
			\end{align*}
			\label{rozszerzonymodel}
		\end{figure}
		Gdzie $\Delta y = y_k - C x_k $ reprezentuje (obserwowane bezpośrednio) na wyjściu obiektu skutki
		awarii torów pomiarowych – różnicę pomiędzy zmierzonymi na wyjściu wartościami sygnałów
		wyjściowych, a wartościami wynikającymi z aktualnego, rzeczywistego stanu obiektu.

		Równanie obserwatora pełnego rzędu (dim x = n) dla modelu rozszerzonego \ref{rozszerzonymodel} przyjmuje
		postać:
		\begin{figure}[H]
			\begin{align*}
				x^E_{k+1} = [A - LC]x^E_k + Ly_k + Bu_k + Ed_k + Ff_k \\ 
				\Delta y = y_k - Cx^E_k
			\end{align*}
			\label{pelnyobserwator}
		\end{figure}
		Warto zauważyć, że rozszerzenie modelu obiektu nie narusza jego ewentualnej
		wykrywalności. Nie narusza bowiem obserwowalności układu (macierze $A$, $C$ nie uległy
		zmianie w wyniku rozszerzenia modelu) oraz nie zmienia rozkładu wartości własnych
		macierzy $[A - LC]$.
	\section{Obserwator diagnostyczny}
		Celem obserwatora jest w tym przypadku wykrywanie uszkodzeń w systemie. Zastosowanie
		banku obserwatorów diagnostycznych ponadto pozwala zlokalizować uszkodzenie.
		Aktualny błąd estymacji stanu $e_k$ jest równy:
		\begin{equation}
			e_k = x_k - x^E_k
		\end{equation}
		Tym samym residuum jest równe 
		\begin{equation}
			r_k  = C e_k - \Delta y
		\end{equation}
		Odejmując równania \ref{rozszerzonymodel} oraz \ref{pelnyobserwator} otrzymujemy
		równanie dynamiki błędu estymacji stanu:
		\begin{equation}
			e_{k+1} = [ A - LC ] e_k + Bu_k + Ed_k + Ff_k
			\label{bladestymacjistanu}
		\end{equation}
		Jeżeli macierz $A – LC$ jest stabilna (jej wartości własne leżą we wnętrzu okręgu
		jednostkowego na płaszczyźnie zespolonej), to oznacza to, że obserwator jest stabilny.
		Jego wektor stanu zmierza asymptotycznie do punktu równowagi. Po zaniknięciu
		przebiegów przejściowych związanych z początkową nieznajomością stanu obiektu, o ile
		w systemie nie wystąpiła jak dotąd awaria i nie był on zaburzany przez zakłócenia, to na
		podstawie \eqref{bladestymacjistanu} można stwierdzić, że błąd estymacji stanu obiektu asymptotycznie
		zmierza do zera. Awaria lub zakłócenie zaburzają osiągnięty przez system stan
		równowagi.

		Stad pojawienie się niezerowych próbek błędu estymacji w stanie ustalonym, świadczy o
		pojawieniu się jednego lub obu czynników zaburzających. Ponieważ w modelu obiektu
		ujęto zakłócenia mierzalne, to tę grupę czynników można zidentyfikować na podstawie
		ich aktualnego pomiaru stwierdzając wystąpienie lub nie wystąpienie zaburzenia. Tym
		samym pozwala związać przyczynę zaburzenia z ewentualną awarią, albo z zakłóceniem
		niemierzalnym.

		To spostrzeżenie jest podstawa metody detekcji awarii, opartej o zastosowanie
		obserwatora rozszerzonego modelu obiektu. Awaria, ze względu na swój skokowy
		charakter powoduje nieciągłość rzeczywistej zmiennej stanu. Jeżeli jest to zmienna
		mierzalna, to wykrycie takiej nieciągłości (ew. przekroczenia pewnej wartości progowej
		przez jej aktualną różnicę skończoną 1-go rzędu) jest dobrą przesłanką do
		wnioskowania o awarii, ale nie wskazuje konkretnego źródła awarii.

		Zastosowanie w tym celu banku obserwatorów, z których jest każdy oparty o model
		obiektu rozszerzonego o model konkretnej, pojedynczej awarii pozwala daną awarię
		zlokalizować.

		Niestety, w estymatorze stanu (obserwatorze, który rekonstruuje niemierzalne zmienne
		stanu), nieciągłości estymaty ujawnią się z pewnym opóźnieniem (zależnym od dynamiki
		obserwatora – rozkładu wartości własnych jego macierzy stanu: $A – LC$). Biorąc pod
		uwagę szybkość współczesnych mikrokontrolerów, pomijamy znikome opóźnienie
		związane z czasem obliczenia kolejnej estymaty, tak więc jest to problem związany z
		oczekiwaniem na kolejne dane, wykorzystywane do obliczeń na bieżąco.
		Często niemierzalne zmienne stanu charakteryzują się dużą dynamiką. Wówczas
		pewnym ułatwieniem detekcji może być użycie obserwatora zredukowanego, który
		powinien charakteryzować się również szybką dynamiką, co powinno skrócić czas od
		wystąpienia awarii do jej wykrycia poprzez estymatę. Inną możliwość detekcji daje
		śledzenie nieciągłości sekwencji wyjściowych.
	\section{Synteza obserwatora zredukowanego}
		Obserwator zredukowany, albo obserwator niepełnego rzędu, znajduje zastosowanie w
		sytuacji, gdy część zmiennych stanu obiektu (zmienne wewnętrzne) nie może być zmierzona
		bezpośrednio, ale wymaga rekonstrukcji na podstawie znajomości modelu obiektu oraz
		danych wejścia-wyjścia.
		Podzielmy wektor stanu obiektu na dwa podwektory
		\begin{equation}
			x_1, x_2: x = [ x_1, x_2] 
		\end{equation}
		, przy czym w $x_1$
		pogrupowano dostępne pomiarowo zmienne stanu układu, a w $x_2$ - pozostałe.
		\begin{equation}
			dim(x) = n, dim(x_2) = l, dim(x_1) = n-l.
		\end{equation}
		
		Przestrzeń stanu $X$ można podzielić na dwie podprzestrzenie $X_1$ oraz $X_2$, których suma
		prosta jest równa $X$:
		\begin{equation }
			X = X_1 ⊕ X_2
		\end{equation}
\end{document}
