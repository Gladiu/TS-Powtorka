\author{Kasiński}
\title{Wykład 7}
\documentclass{article}
\usepackage[margin=1.4cm]{geometry}
\usepackage[utf8]{inputenc}
\usepackage{babel}
\usepackage{polski}
\usepackage{float}
\usepackage{graphicx}
\usepackage{amsmath}
\usepackage{hyperref}
\begin{document}
	\maketitle
	\section{Porównanie metod linearyzacji}
		\subsection{Przykład 1}
			Rozważmy system dynamiczny opisany nieliniowymi równaniami stanu
			\begin{align*}
				\dot{x_1} = x_1 cos(x_2) + (x_2)^2 \\
				\dot{x_2} = x_1 sin(x_2) + x_1 + x_2 + (x_1)^2 
			\end{align*}
			Jego nieliniowość jest istotna. Ma charakter dynamiczny, czyli istnieje zależność
			prędkości zmian stanu od
			funkcji okresowych oraz wyższych potęg zmiennych stanu. Oba podsystemy są nieliniowo
			skrośnie posprzęgane. Obecność funkcji okresowych powoduje
			występowanie przeliczalnej liczby izolowanych punktów stacjonarnych. Model
			nieliniowy nie jest odwracalny, czyli nie istnieje funkcja $f^{-1}(x)$,
			odwrotna do $f(x)$
			zatem możliwa jest jedynie jego przybliżona linearyzacja . Zbadamy jego właściwości w
			otoczeniu punktu $x = [ 0 \space 0 ]^T $
			
			Łatwo sprawdzić że punkt $x = [ 0 \space  0 ]^T$  jest punktem stacjonarmy układu, czyli
			dla tego punktu prawdziwa jest rówość $\dot{x} = 0$. Taki punkt jest już podstawą
			do linearyzacji lokalnej.

			Podejście bezpośrednie polega na "zlinearyzowania" składników prawych stron zakładając, że
			pozostajemy we wnętrzu hiperkuli o małym promieniu $\varepsilon$. Pomijamy „małe” wyższych
			rzędów. I tak: $cos \varepsilon \approx 1$, a więc $x_1 cos \varepsilon \approx x1$
			, $(x_2)^2 = \varepsilon \cdot 2$, $sin \varepsilon \approx 0$, a więc $x1 sin \varepsilon \approx 0$, $(x_1)^2 = \varepsilon \cdot 2$ .
			Po pominięciu nieistotnych składników otrzymujemy zlinearyzowany model postaci:
			\begin{align*}
				\dot{x_1} = x_1 \\
				\dot{x_2} = x_1 + x_2 
			\end{align*}
			Z powyższych równań otrzymujemy macierz A
			\begin{equation}
				A = 
				\begin{pmatrix}
					1 & 0 \\
					1 & 1 
				\end{pmatrix}
			\end{equation}

			Możemy też zastosować bardziej uniwersalną metodę, czyli metodę Jakobianową

			\begin{equation}
				\frac{df}{dx^T}
				\begin{pmatrix}
					cos(x_2) & 2x_2 - x_1 sin(x_1) \\
					2x_1 + sin(x_2) + 1 & 1 + x_1 cos(x_2)
				\end{pmatrix}
			\end{equation}
			Podstawiając $x = [ 0 \space 0 ]^T$ do Jakobianu otrzymujemy macierz $A$ taką jak wcześniej
			\begin{equation}
				A = 
				\begin{pmatrix}
					1 & 0 \\
					1 & 1 
				\end{pmatrix}
			\end{equation}
		\subsection{Przykład 2}
			Równanie wejścia wyjścia obiektu sterowania ma postać
			\begin{equation}
				\ddot{x} + 4(\dot{x})^2 + x^3u = 0
			\end{equation}
			Będziemy linearyzować powyższe równanie wokół punktu stacjonarnego $x = 0$
			Linearyzując ją metodą bezpośrednią, czyli pomijając i szacując składniki
			otrzymujemy 
			\begin{equation}
				\ddot{x} + u = 0
			\end{equation}
			Z powyższego równania otrzymujemy elegancki model w przestrzeni stanu
			\begin{align*}
				\dot{x_1} = x_2 \\
				\dot{x_2} = -u \\
			\end{align*}
			Skąd mamy równie eleganckie macierze 
			\begin{equation}
				A = 
				\begin{pmatrix}
					0 & 1 \\
					0 & 0 
				\end{pmatrix}
			\end{equation}
			oraz
			\begin{equation}
				B = 
				\begin{pmatrix}
					0 \\
					-1 
				\end{pmatrix}
			\end{equation}
			Astatyzm – cecha wykazywana przez obiekty i układy z co najmniej jednym
			pierwiastkiem równania charakterystycznego o części rzeczywistej równej
			zero. W zależności od krotności tego pierwiastka można mówić o astatyzmie
			n-tego rzędu.
			\newline
			\newline
			Model zlinearyzowany wskazuje na lokalną obecność astatyzmu drugiego rzędu 
			w otoczeniu punktu stacjonarnego $x = [0 0]^T$
			Równanie opisuje ruch ciała o masie m=1, w środowisku silnych (w kwadracie) oporów
			ruchu pod wpływem rozciągania (u jest odkształceniem) nieliniowej sprężyny.
			Sygnałem wyjściowym jest przesunięcie środka masy układu. Linearyzacja metodą
			jacobianową prowadzi do identycznego modelu lokalnego.
	\section{Liniowe modele kanoniczne w przestrzeni stanu}
		W przypadku obiektu typu SISO, definiując zmienne stanu jako zmienne fazowe
		otrzymaliśmy, w przypadku braku zer w transmitancji układu, model, w którym A –
		macierz (stanu) systemu miała formę Frobeniusa:
		\begin{equation}
			A=\begin{pmatrix}
			  0    &   1    &   0    &   0    & \cdots &   0      &   0    \\
			  0    &   0    &   1    &   0    & \cdots &   0      &   0    \\
			  0    &   0    &   0    &   1    & \cdots &   0      &   0    \\
			  0    &   0    &   0    &   0    & \cdots &   0      &   0    \\
			\vdots & \vdots & \vdots & \vdots & \ddots & \vdots   & \vdots \\
			  0    &   0    &   0    &   0    & \cdots &   0      &   1    \\
			 -a_{1}&  -a_{2}&  -a_{3}&  -a_{4}& \cdots & -a_{m-1} & -a_{m}
			\end{pmatrix}
		\end{equation}
		Natomiast 
		\begin{equation}
			B = 
			\begin{pmatrix}
				0 \\
				0 \\
				0 \\
				\vdots \\
				0 \\
				k 
			\end{pmatrix}
		\end{equation}
		Gdzie k jest współczynnikiem wzmocnienia, a 
		\begin{equation}
			C = 
			\begin{pmatrix}
				1 & 0 & 0 & \cdots & 0
			\end{pmatrix}
		\end{equation}
		A $D=0$
 
		W przypadku , gdy A jest wysokiego rzędu i wysoką wartość wymiarów n, wtedy
		powyższe równania reprezentują postać kanoniczną modelu liniowego obiektu SISO
		\newline
		W przypadku obiektu typu MIMO model ma postać macierzy blokowych,
		gdzie poszczególne bloki są modelami odpowiednich par wejścia-wyjścia
		Jeżeli w transmitancji obiektu SISO występują zera, to kanoniczna postać fazowa ma identyczną
		macierz stanu A i macierz wyjścia C jak w przypadku bez zer, $B = [ \beta_1, \beta_2 \ddots \beta_n]^T$ , $D = [ \beta_o ]$, gdzie
		$\beta_i (i=0, 1, . . . , n)$ są współczynnikami ze zmodyfikowanych dla
		tego przypadku definicji
		fazowych zmiennych stanu. Zwraca uwagę obecność sprzężenia transmisyjnego (
		bezpośredniego sprzężenia „do przodu” ) od wejścia do wyjścia obiektu. Również
		charakterystyczny jest bezpośredni, proporcjonalny wpływ sygnału sterującego na każdą
		ze zmiennych stanu.
		\subsection{Wyprowadzenie modelu odsprzężonego}
			Inną postacią kanoniczną modelu jest tzw, model odsprzężony, którego macierz A jest
			macierzą diagonalną:
			\begin{equation}
				A=\begin{pmatrix}
				  \lambda_1    &   0    &   0    &   0    & \cdots &   0    &   0      \\
				  0    &   \lambda_2    &   0    &   0    & \cdots &   0    &   0      \\
				  0    &   0    &   \lambda_3    &   0    & \cdots &   0    &   0      \\
				  0    &   0    &   0    &    \lambda_4   & \cdots &   0    &   0      \\
					\vdots & \vdots & \vdots & \vdots & \ddots & \lambda_{n-1} & \vdots   \\
				  0    & \vdots & \vdots & \vdots & \ddots &   0    & \lambda_n\\
				\end{pmatrix}
			\end{equation}

			$\lambda_i$ są wartościami własnymi danej macierzy stanu A (uzyskanej z opisu obiektu, na ogół niediagonalnej).
			Inne macierze wyglądają wtedy następująco	
			\begin{equation}
				B = 
				\begin{pmatrix}
					1 \\
					1 \\
					1 \\
					\vdots \\
					1 \\
					1 
				\end{pmatrix}
			\end{equation}
			\begin{equation}
				C = 
				\begin{pmatrix}
					c_1 & c_2 & c_4 & \cdots & c_n
				\end{pmatrix}
			\end{equation}
			natomiast $D = [c_0]$

			W tym miejscu założyliśmy, że dany układ o liniowym modelu stanowym $<A,B,C,D>$ ma
			transmitancję operatorową w postaci zespolonej funkcji wymiernej, tj. ilorazu
			wielomianów o współczynnikach rzeczywistych, zmiennej zespolonej $s$.
			\begin{equation}
				G(s) = \frac{L(s)}{M(s)}
			\end{equation}
			Dodatkowo zakładamy, że wielomian w mianowniku ma pojedyncze pierwiastki 
			oraz jest ich $n$ i można wykazać,
			że są one tożsame z wartościami własnymi macierzy A. Ponadto zakładamy, że zera
			transmitancji mają wartości różne od wartości biegunów.
			W przypadku wielomianów zmiennej zespolonej ze współczynnikami rzeczywistymi jego
			pierwiastkami mogą być liczby rzeczywiste lub zespolone. Wielomian stopnia
			nieparzystego ma co najmniej jeden pierwiastek rzeczywisty. Jeżeli pierwiastkiem jest
			liczba zespolona, to jest nim również liczba zespolona sprzężona z nią

			Załóżmy, że wyznaczyliśmy wartości pierwiastków wielomianów L(s) i M(s) i wynoszą
			one odpowiednio  $[z_1 z_2 \ddots z_m ]$ i $[\lambda_1 \lambda_2 \ddots \lambda_n ]$
			, gdzie $m$ jest stopniem wielomianu w
			liczniku i $m \leq n$. Wówczas możemy przedstawić transmitancję w postaci
			sfaktoryzowanych wielomianów:
			\begin{equation}
				G(s) = \frac{k(s-z_1)(s-z_2)}{(s-\lambda_1)(s-\lambda_2)}\cdots\frac{(s-z_m)}{s-\lambda_m)}\cdots\frac{1}{(s-\lambda_n)}
			\end{equation}
			Można dokonać rozkładu powyższego wyrażenia na ułamki proste:
			\begin{equation}
				\frac{Y(s)}{U(s)} = \frac{L(s)}{M(s)} = c_0 + \sum^n_{i=0}\frac{c_i}{s-\lambda_i}
			\end{equation}
			Każdy element pod znakiem sumy można interpretować jako elementarną transmitancję
			elementu 1-go rzędu, a całe wyrażenie po prawej stronie jako połączenie równoległe tych
			członów (sygnałem wejściowym do każdego z nich jest u(t), a sygnałem wyjściowym y(t)
			jest suma ich sygnałów wyjściowych. Układ równań stanu tak przekształconego obiektu
			przedstawia się następująco:
			\begin{align*}
				\dot{x_1} = \lambda_1 x_1 + u \\
				\dot{x_2} = \lambda_2 x_2 + u \\
				\cdots \\
				\dot{x_n} = \lambda_n x_n + u \\
			\end{align*}
			Stałe $c_0 c_1 \ddots c_n$ można wyznaczyć jako funkcje współczynników $a_i$
			oraz $b_i$ wielomianów $L(s)$ oraz $M(s)$, porównując porównując współczynniki
			przy odpowiednich potęgach $s$ po obu stronach tożsamości.
			Ta interpretacja uzasadnia postać macierzy modelu odsprzężonego przytoczone powyżej.
			W przypadku, gdy bieguny transmitancji są wielokrotne,
			nia da się zdiagonalizować modelu. Można jedynie uzyskać postać blokowo diagonalną,
			gdzie bloki przyjmują postać tzw. klatek Jordana.
			Przy biegunie dwukrotnym $\lambda_i$ taka klatka ma postać 
			\begin{equation}
				\begin{pmatrix}
					\lambda_i & 0 \\
					0 & \lambda_i
				\end{pmatrix}
			\end{equation}
			Natomiast przy biegunie trzykrotnie mamy
			\begin{equation}
				\begin{pmatrix}
					\lambda_i & 0 & 0 \\
					0 & \lambda_i & 0 \\
					0 & 0 & \lambda_i
				\end{pmatrix}
			\end{equation}
			Równania stanu o indeksach $i, i+1$ mają wtedy postać
			\begin{align*}
				\dot{x_i} = \lambda_ix_i + x_{i+1} \\ 
				\dot{x_{i+1}} = \lambda_ix_{i+1} + u
			\end{align*}
	\section{Geometryczna interpretacja liniowego modelu stanowego.}
		Dla większej przejrzystości poniższych rozważań przypomnę, że płaszczyzny
		są w geometrii. Gdy mówimy o wymiarach macierzowych i jakieś równanie określa
		płaszczyznę w przestrzeni euklidesowej n-wymiarowej to mówimy o 
		hiperpłaszczyźnie.
		Przestrzeń euklidesowa to taka przestrzeń jak jest w prawdziwym życiu.
		Przypomnijmy sobie również definicję układu zdegenerowanego. 
		To taki układ którego punkty stacjonarne nie 
		stanowią izolowanych punktów, a są na przykład krzywą.
		\newline
		Przejdźmy do karuzeli spierdolenia fundowaną przez kasińskiego.
		\newline
		Przestrzeń stanu jest przestrzenią wielowymiarową
		$X = R^n$ , gdzie $n$ jest wymiarem wektora stanu.

		W przypadku układu niezdegenerowanego, możemy wyobrazić sobię trajektorię stanu
		jako parametryczną krzywą $n$-wymiarową w przestrzeni $X$, gdzie parametrem
		tej krzywej jest czas.
		Elementami tej trajektorii są prędkości zmian stanu.

		Równanie stanu opisujące model dynamiki, w sensie geometrycznym opisuje
		rzutowanie wektora stanu $x(t)$ na hiperpłaszyznę styczną do trajektorii stanu.

		Jest to hiperpłaszyzna styczna w jednym punkcie. Ponieważ znając wektor stanu $x(t)$ oraz 
		sterowań $u(t)$, nie jesteśmy w stanie otrzymać całej trajektorii.
		Aby to udowodnić wystarczy spojrzeć na poniższe równanie
		\begin{equation}
			\dot{x} = Ax + Bu
		\end{equation}

		To równanie również opisuje analogiczne rzutowanie wektora sterowań $u(t)$ w tę
		hiperpłaszczyznę. Inaczej rzecz się ma w przypadku
		modeli z czasem dyskretnym. Tutaj różnicowe równanie stanu opisuje rzutowanie w
		przestrzeń stanu (w rozmaitość różniczkową stanu , gdy model jest nieliniowy).


		Macierze $A$ i $B$ modelu liniowego można interpretować jako operatory rzutowania
		odpowiednio elementów przestrzeni stanu i przestrzeni
		sterowań w inną przestrzeń, taką jak hiperpłaszczyzna styczna czy rozmaitość.
		Podobnie macierze modelu wyjścia $C$ i $D$ można
		interpretować jako operatory rzutowe odpowiednio wektorów stanu i wektorów
		sterowań w $p$ wymiarową przestrzeń wyjść. Należy pamiętać, że wektor stanu jest
		wektorem wodzącym, natomiast wektor sterowań, w odniesieniu do układu
		współrzędnych stanu jest wektorem swobodnym. W przypadku modeli liniowych
		przestrzeń stanu jest przestrzenią liniową, a hiperpłaszczyzna styczna jest 
		\href{https://en.wikipedia.org/wiki/Isomorphism}{izomorficzna}
		z przestrzenią stanu (jest obrócona względem przestrzeni stanu wskutek działania na
		każdy jej element przez macierz $A$).

		Macierz stanu A działając na chwilowy wektor stanu $x(t)$ rzutuje go na hiperpłaszczyznę
		styczną do trajektorii w punkcie $x(t)$. Podobnie macierz sterowań $B$ rzutuje chwilowy
		wektor sterowań $u(t)$ w tę hiperpłaszczyznę. Efekt mnożenia wektora wodzącego przez
		macierz pełnego rzędu $A$ jest następujący:
		\begin{itemize}
			\item Wektor zmienia kierunek czyli ulega złożonemu obrotowi
				w przestrzeni $n$-wymiarowej
			\item wektor zmienia swoją długość w zależności od wartości $det A$.
				Dla $det A =1$ nie
				zmienia długości, dla $det A > 1 $ulega wydłużeniu,
				dla $det A < 1$, skróceniu
			\item addytywny składnik zależny od wektora sterowania powoduje translację końca
				wektora stanu w podprzestrzeni $m$-wymiarowej o ile kolumny macierzy B są
				liniowo niezależne.
		\end{itemize}
		\subsection{Diagonalizacja macierzy stanu $A$}
			Jeszcze przejrzyściej jest to widoczne w przypadku modelu liniowego z czasem
			dyskretnym, gdzie te działania odwzorowują wektor stanu na przestrzeń stanu.
			Są takie kierunki w przestrzeni stanu, dla których działanie w postaci mnożenia wektora
			o takim kierunku przez macierz $A$ nie powoduje zmiany kierunku wektora, a jedynie
			zmianę jego długości. Wyznaczające te kierunki wektory noszą nazwę wektorów
			własnych. Macierz $n \times n$ pełnego rzędu ma n wektorów własnych. Są one określone co do
			kierunku, ich długość jest nieokreślona. Tworzą układ wektorów liniowo niezależnych
			Oznaczmy je jako $e^1, e^2, \cdots ,e^n$ .
			Macierz, której kolumny stanowią wektory własne nazywamy
			macierzą modalną:
			\begin{equation}
				M = 
				\begin{pmatrix}
					e^1 & e^2 & \cdots & e^n
				\end{pmatrix}
			\end{equation}
			Macierz modfalna jest nieosobliwa (można ją odwrócić). Macierz $A$ ma tyle macierzy
			modalnych ile jest permutacji kolumn macierzy $M$ czyli $n!$ Mówimy, że jest ona określona
			z dokładnością do permutacji.
			Wektory własne spełniają następujące równania:
			\begin{equation}
				Ae^i = \lambda_i e^i
			\end{equation}
			Dla $i= 1,2,\ddots n$
			Jest to macierzowe równanie liniowe
			\begin{equation}
				[A_i-\lambda_iI]e^i = 0
			\end{equation}
			W zapisie macierzowym można je przedstawić następująco
			\begin{equation}
				A [ e^1 e^2  \cdots  e^n ] = [ e^1  e^2  \cdots  e^n ] \Lambda
			\end{equation}
			gdzie $\Lambda = diag[ \lambda^1  \lambda^2 \ddots  \lambda^n ]$ więc
			\begin{align*}
				AM = M\Lambda \\ 
				\Lambda = M^{-1}AM
			\end{align*} 
			W ten sposób dokonuje się diagonalizacji macierzy stanu $A$
	\section{Równoważność liniowych modeli w przestrzeni stanu}
		Załóżmy, że sporządzono dwa modele systemu
		w przestrzeni stanu dokonując odmiennego wyboru zmiennych stanu:
		\newline
		Model I - wektor stanu $x$:
		\begin{align*}
			\dot{x(t)} = Ax(t) + Bu(t) \\
			y(t) = Cx(t) + Du(t)
		\end{align*}
		Model II - wektor stanu $z$:
		\begin{align*}
			\dot{z(t)} = A^*z(t) + B^*u(t) \\
			y(t) = C^*z(t) + D^*u(t)
		\end{align*}
		Spełniony jest warunek $dim(x)=dim(z)=n$
		Oraz transmitancje operatorowe obu modeli są identyczne
		\begin{equation}
			C[sI-A]^{-1}B+D = C^*[sI-A^*]^{-1}B^*+D^*
		\end{equation}
		\subsection{Warunek równoważności liniowych}
			Na to by modele I i II były równoważnymi opisami tego samego obiektu sterowania,
			wystarczy istnienie nieosobliwej macierzy $P$ przekształcenia
			wektora stanu
			$x$ w wektor stanu $z$ zwana macierzą podobieństwa 
			\begin{equation}
				z = Px
			\end{equation}
			Podstawiając do drugiego modelu za $z$ oraz $\dot{z}$
			powyższe związki otrzymujemy:
			\begin{align*}
				P\dot{x(t)} = A^*Px(t) + B^*u(t) \\
				y(t) = C^*Px(t) + D^*u(t)
			\end{align*}
			Jeżeli oba modele są równoważne to muszą istnieć zależności
			\begin{align*}
				A = P^{-1}A^*P \\
				A^* = P^{-1}AP 
			\end{align*}
			\begin{align*}
				B = P^{-1}B^*P \\
				B^* = P^{-1}BP 
			\end{align*}
			\begin{align*}
				C = P^{-1}C^*P \\
				C^* = P^{-1}CP 
			\end{align*}
			\begin{align*}
				D = P^{-1}D^*P \\
				D^* = P^{-1}DP 
			\end{align*}
			Jeżeli za macierz podobieństwa $P$ przyjąć macierz 
			modalną $M$ danej macierzy stanu $A$, to
			można dany model zdiagonalizować 
			Mamy wtedy przypadek jednokrotnych biegunów transmitancji
			- wartości własnych macierzy A,
			Jeśli nie jest możliwa diagonalizacja to można nadal w
			istotnym stopniu zlikwidować sprzężenia
			skrośne w modelu obiektu jest to wtedy przypadek wielokrotnych biegunów transmitancji - wartości własnych macierzy A.
			Usunięcie sprzężeń skrośnych ułatwia projekt struktury układu sterującego.
\end{document}
