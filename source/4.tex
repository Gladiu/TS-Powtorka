\author{Kasiński}
\title{Wykład 3}
\documentclass{article}
\usepackage[utf8]{inputenc}
\usepackage{babel}
\usepackage{polski}
\usepackage{float}
\usepackage{graphics}
\maketitle
\begin{document}
	Ważnym pojęciem w analizie właściwości obiektów sterowania opisanych nieliniowymi
	równaniami stanu i wyjścia jest punkt stacjonarny. Rozważmy następujący model:

	\begin{equation}
		\dot{x}(t) = f(x(t), u(t))
	\end{equation}
	\begin{equation}
		y(t) = h(x(t))
	\end{equation}

	Pierwsze równanie, to wektorowe równanie stanu określające, wymuszoną przez
	sygnały sterujące { $ u_i(t), i=1,...,m $ }, trajektorię stanu układu dynamicznego w przestrzeni
	n-wymiarowej jako symultaniczne rozwiązanie układu równań różniczkowych pierwszego rzędu postaci:

	\begin{equation}
		\dot{x_1} = f_1(x_1(t), x_2(t),...,x_n(t); u_1(t),...,u_m(t))
	\end{equation}
	\begin{equation}
		\dot{x_2} = f_1(x_1(t), x_2(t),...,x_n(t); u_1(t),...,u_m(t))
	\end{equation}

	Aż do

	\begin{equation}
		\dot{x_n} = f_1(x_1(t), x_2(t),...,x_n(t); u_1(t),...,u_m(t))
	\end{equation}

	Wystarczy, że jedna z funkcji $f_i$ jest nieliniowa, to mamy zadanie dynamiki nieliniowej, w
	którym to przypadku stosunkowo rzadko można uzyskać rozwiązanie jawne. Można
	wówczas spróbować je rozwiązać metodą całkowania numerycznego, pod warunkiem że
	rozwiązanie istnieje a iteracje są do niego zbieżne. Trudno jednak udowodnić istnienie
	rozwiązania w większości przypadków wielowymiarowych problemów nieliniowych,
	zwłaszcza dla przypadków z wymuszeniem.

	Dlatego dostępnym sposobem analizy zachowania układu w niewielkim otoczeniu
	wybranego punktu przestrzeni stanu jest lokalna linearyzacja modelu. Autonomiczny,
	nieliniowy obiekt dynamiczny opisany jest równaniami:

	\begin{equation}
		\dot{x}(t) = f(x(t))
	\end{equation}

	\begin{equation}
		y(t) = h(x(t))
	\end{equation}

	W szczególności istotne jest zachowanie układu autonomicznego (inaczej swobodnego,
	czyli pozbawionego sterowania) w otoczeniu tzw. punktów stacjonarnych, a więc takich,
	w których układ pozostaje w stanie ustalonym. Punkty te można wyznaczyć rozwiązując
	układ algebraicznych równań nieliniowych postaci:
	
	\begin{equation}
		f(x(t)) = 0
	\end{equation}


\end{document}
