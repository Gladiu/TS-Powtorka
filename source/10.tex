\author{Kasiński}
\title{Wykład 10}
\documentclass{article}
\usepackage[margin=1.4cm]{geometry}
\usepackage[utf8]{inputenc}
\usepackage{babel}
\usepackage{polski}
\usepackage{float}
\usepackage{graphicx}
\usepackage{amsmath}
\usepackage{hyperref}
% More defined colors
\usepackage[dvipsnames]{xcolor}
% Required package
\usepackage{tikz}
\usetikzlibrary{positioning}
\begin{document}
	\maketitle
	\section{Synteza obserwatora dla rozszerzonego modelu obiektu}
		Do konstrukcji algorytmu obserwatora wykorzystuje się model dyskretny (liniowy)
		obiektu sterowania:
		\begin{align*}
			x_{k+1} = Ax_k + Bu_k \\ 
			y_k = C x_k + Du_k
		\end{align*}
		Obserwator rekonstruuje wartość wektora stanu $x_{k-N}$
		na podstawie danych, w postaci zapamiętanych próbek $y_i$ oraz $u_i$,
		\begin{equation}
			i=k-N, k-N+1, \ldots , k
		\end{equation}
		Równanie różnicowe obserwatora pełnego ma postać:
		\begin{align*}
			z_{k+1} = [A-LC] z_k - Ly_k + Bu_k \\ 
			x^E_k = z_k
		\end{align*}
		gdzie wskaźnik górny $^E$ wskazuje, że jest to estymata wektora stanu.
		Zdefiniujmy wektor residuum układu w następujący sposób:
\end{document}
