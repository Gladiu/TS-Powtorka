\author{Kasiński}
\title{Wykład 3}
\documentclass{article}
\usepackage[utf8]{inputenc}
\usepackage{babel}
\usepackage{polski}
\usepackage{float}
\usepackage{graphicx}
\usepackage{amsmath}
\begin{document}
	\maketitle
	Ważnym pojęciem w analizie właściwości obiektów sterowania opisanych nieliniowymi
	równaniami stanu i wyjścia jest punkt stacjonarny. Rozważmy następujący model:

	\begin{equation}
		\dot{x}(t) = f(x(t), u(t))
	\end{equation}
	\begin{equation}
		y(t) = h(x(t))
	\end{equation}

	Pierwsze równanie, to wektorowe równanie stanu określające, wymuszoną przez
	sygnały sterujące { $ u_i(t), i=1,...,m $ }, trajektorię stanu układu dynamicznego w przestrzeni
	n-wymiarowej jako symultaniczne rozwiązanie układu równań różniczkowych pierwszego rzędu postaci:

	\begin{equation}
		\begin{aligned}
			\dot{x_1} = f_1(x_1(t), x_2(t),...,x_n(t); u_1(t),...,u_m(t)) \\
			\dot{x_2} = f_1(x_1(t), x_2(t),...,x_n(t); u_1(t),...,u_m(t)) \\
			\dot{x_n} = f_1(x_1(t), x_2(t),...,x_n(t); u_1(t),...,u_m(t))
		\end{aligned}
	\end{equation}

	Wystarczy, że jedna z funkcji $f_i$ jest nieliniowa, to mamy zadanie dynamiki nieliniowej, w
	którym to przypadku stosunkowo rzadko można uzyskać rozwiązanie jawne. Można
	wówczas spróbować je rozwiązać metodą całkowania numerycznego, pod warunkiem że
	rozwiązanie istnieje a iteracje są do niego zbieżne. Trudno jednak udowodnić istnienie
	rozwiązania w większości przypadków wielowymiarowych problemów nieliniowych,
	zwłaszcza dla przypadków z wymuszeniem.

	Dlatego dostępnym sposobem analizy zachowania układu w niewielkim otoczeniu
	wybranego punktu przestrzeni stanu jest lokalna linearyzacja modelu. Autonomiczny,
	nieliniowy obiekt dynamiczny opisany jest równaniami:

	\begin{equation}
		\dot{x}(t) = f(x(t))
	\end{equation}

	\begin{equation}
		y(t) = h(x(t))
	\end{equation}

	W szczególności istotne jest zachowanie układu autonomicznego (inaczej swobodnego,
	czyli pozbawionego sterowania) w otoczeniu tzw. punktów stacjonarnych, a więc takich,
	w których układ pozostaje w stanie ustalonym. Punkty te można wyznaczyć rozwiązując
	układ algebraicznych równań nieliniowych postaci:
	
	\begin{equation}
		f(x(t)) = 0
	\end{equation}

	Zakładamy, że układ jest regularny, lub inaczej mówiąc niezdegenerowany. Wówczas rozwiązanie stanowi zbiór
	izolowanych punktów w przestrzeni stanu takich, że 

	\begin{equation}
		x_s^i 
	\end{equation}
	dla 
	\begin{equation}
 		i=1,2,..., r
	\end{equation}
	Liczba miejsc zerowych funkcji
	wektorowej f
	wynika nie tylko z liczby współrzędnych wektora (liczby równań
	algebraicznych), ale zależy od typów nieliniowości.
	\subsection{Przykłady}
		Dany jest układ dynamiczny pierwszego rzędu dany równaniem

		\begin{equation}
			\dot{x} = -x^2 + 2
		\end{equation}

		Rozwiązując równanie kwadratowe otrzymujemy rozwiązania
		$x = \sqrt{2}$ oraz $x = -\sqrt{2}$
		Są to dwa punkty stacjonarne.

		Następny przykład ma bardziej złożone rozwiązanie.

		\begin{equation}
			\begin{aligned}
				\dot{x_1} = x_1 + x_2 \\ 
				\dot{x_2} = x_1 \left( \cos{x_2} + \frac{1}{2} \right)
			\end{aligned}
		\end{equation}

		Ma przeliczalnie wiele punktów stacjonarnych:

		\begin{equation}
			\left[ -\frac{\pi}{3} - 2k\pi, \frac{\pi}{3} + 2k\pi \right]
		\end{equation}

		Dla

		\begin{equation}
			k = 0, 1, ...
		\end{equation}
		
		Wiedząc że układ zdegenerowany to taki którego punkty stacjonarne nie 
		stanowią izolowanych punktów, a są na przykład krzywą. Możemy stwierdzić,
		że takie układy charakteryzują się lokalnym defektem rzędu rozmaitości stanu.
		lub istnieniem podrozmaitości osobliwej.

		\section{Nieliniowe układy afiniczne}

		Ważną podklasę ogólnych modeli liniowych stanowią tak zwane modele afiniczne.

		\begin{equation}
			\dot{x}(t) = f(x(t)) + g_1(x(t))u_1(t) + g_2(x(t))u_2(t) + ...
			+ g_m(x(t))u_m(t) 
		\end{equation}

		\begin{equation}
			y(t) = h(x(t))
		\end{equation}

		Modele tej klasy dobrze reprezentują ( nie tylko lokalnie) dynamikę 
		układów nieliniowych, w których w równaniach stanu, można rozdzielić argumenty
		w postaci zmiennych stanu od sygnałów sterujących. Nie nakładają dodatkowych 
		ograniczeń na dynamikę układu autonomicznego. Akcję sterowań można interpretować
		jako możliwość wpływania na ewolucję wektora stanu w każdym aktualnym punkcie x(t)
		trajektorii, w hiperpłaszczyźnie $T_x(x(t))$ stycznej do niej w tym punkcie.
		
		Pola wektorowe $g_i(x(t))$  dla $i= 1, ... , m$ można
		interpretować jako kierunkowe współczynniki wzmocnienia odpowiednich sygnałów
		sterujących. Podkreślają one lokalny charakter sterowania (zmianę kierunku i „siły działania”
		skalarnego sygnału sterującego wzdłuż trajektorii stanu). Wektor sterowań jest elementem
		liniowej przestrzeni sterowań, co ułatwia operowanie nim. Z punktu widzenia sterowalności
		układu istotne, aby pola wektorowe tworzyły układ wektorów lokalnie liniowo niezależnych,
		a po uzupełnieniu o pole wektorowe $f(x(t))$
		umożliwiały osiąganie dowolnego punktu w
		przestrzeni stanu z dowolnego stanu początkowego poprzez odpowiednie sterowanie.
\end{document}
