\author{Kasiński}
\title{Wykład 9}
\documentclass{article}
\usepackage[margin=1.4cm]{geometry}
\usepackage[utf8]{inputenc}
\usepackage{babel}
\usepackage{polski}
\usepackage{float}
\usepackage{graphicx}
\usepackage{amsmath}
\usepackage{hyperref}
% More defined colors
\usepackage[dvipsnames]{xcolor}
% Required package
\usepackage{tikz}
\usetikzlibrary{positioning}
\begin{document}
	\maketitle
	\section{Badanie obserwowalności obiektu}
		Dysponując adekwatnym modelem analitycznym obiektu sterowania można w oparciu
		o ten model zbadać jego obserwowalność.
		Zakładając liniowość i stacjonarność modelu obiektu sterowania mamy:
		\begin{align*}
			\dot{x} = Ax(t)+Bu(t) \\
			y = Cx(t)+Cu(t) 
		\end{align*}
		W przypadku sterowalności istotne było wykazanie zależności
		wektora stanu od sygnałów sterujących działających w pewnym przedziale czasu.

		Niech $n$ będzie rzędem układu. Wtedy 
		dla przypadku modelu z czasem dyskretnym, do wyznaczenia sterowalności
		chodziło o stwierdzenie, że kierunki
		działania wektora sterowań w kolejnych $n$ chwilach próbkowania 
		rozpinają pełną, $n$-wymiarową przestrzeń wektorową. Spełnienie takiego warunku gwarantuje
		osiągnięcie dowolnego stanu docelowego z dowolnego stanu początkowego w wyniku
		właściwego doboru sekwencji decyzji sterujących.

		Model obiektu reprezentuje powiązania "przyczyna-skutek" pomiędzy funkcjami czasu:
		\begin{equation}
			u(t)|^{t_k}_{t_0} \rightarrow x(t)|^{t_k}_{t_0} \rightarrow y(t)|^{t_k}_{t_0} 
		\end{equation}

		(w układzie z czasem ciągłym), albo szeregami czasowymi:
		\begin{equation}
			{u_1 u_2 \ldots u_{N-1}} \rightarrow{x_1 x_2 \ldots x_{N}} \rightarrow {y_1 y_2 \ldots y_{N}} \rightarrow
		\end{equation}
		w modelu dyskretnym.

		W przypadku sterowalności istotne jest powiązanie
		\begin{equation}
			u(t)|^{t_k}_{t_0} \rightarrow x(t)|^{t_k}_{t_0} 
		\end{equation}

		Jednak jeżeli cel sterowania sformułowano w odniesieniu do trajektorii wyjściowej
		będącą pełnowymiarową krzywą w przestrzeni wyjść $Y^p$, to mówimy raczej o problemie
		osiągalności zadanego punktu docelowego pracy ustalonej $y_k= y(t_k)$ z początkowego
		punktu pracy ustalonej $y_0= y(t_0)$. W takim przypadku warunek rzędu przyjmuje postać:
		\begin{equation}
			rank(C \times S) = p
		\end{equation}

		gdzie: $C$ jest macierzą wyjść obiektu, $S$ jest jego macierzą sterowalności, $p$ -liczbą wyjść.
		Ponadto, $x_0$ takie, że spełnia warunek 
		\begin{equation}
			y_0 = C x_0+ D u_0
		\end{equation}
		oraz $x_K$ takie, że spełnia warunek
		\begin{equation}
			y_K = C x_K+ D u_K
		\end{equation}
		powinny być stanami stacjonarnymi.

		Ustalony punkt pracy winien się charakteryzować stacjonarnością wartości sygnałów
		wyjściowych czyli brakiem zjawisk przejściowych, co najwyżej występowaniem fluktuacji
		wokół stałych wartości na wyjściu, powodowanych zakłóceniami losowymi.
		Stałość sygnałów na wyjściu obiektu nie jest jednak równoznaczna ze stałością stanu
		obiektu.
		Aktywna stabilizacja poziomów sygnałów wyjściowych, związana z wymogiem
		pozostawania obiektu w danym punkcie pracy. Punkt ten jest określany zazwyczaj na podstawie
		charakterystyk statycznych obiektu i wymaga wewnętrznych oddziaływań
		dynamicznych, np. kompensujących zakłócenia procesu. W takim przypadku mówimy, że 
		zmienne stanu są wówczas aktywne dynamicznie.
		Potocznie obserwację jakiejś wielkości np. pewnej zmiennej stanu utożsamia się z
		pomiarem tej wielkości. Tak więc obserwowalność
		niesłusznie utożsamia się z możliwością przeprowadzenia pomiaru.
		Jest to skojarzenie nieuprawnione, ponieważ obserwowalność jest cechą
		obiektu, głębszą niż mierzalność. Obserwowalność oznacza bowiem możliwość
		dokonania obserwacji wartości również tych zmiennych stanu, które nie mogą być
		bezpośrednio zmierzone, na przykład z powodu braku sensora lub jego uszkodzenia.
		Mówimy w takim przypadku o odtwarzaniu lub oszacowaniu takich wartości.
		Oszacowanie jest dokonywane na podstawie dostępnej informacji o obiekcie. Tymi
		danymi są model matematyczny obiektu oraz zarejestrowane przebiegi sygnałów na
		wejściach obiektu wraz z wywołanych przez nie sygnałami odpowiedzi na jego wyjściach
		Warunkami uzyskania oszacowania niemierzalnych zmiennych stanu są:
		\begin{itemize}
			\item obserwowalność obiektu
			\item dostatecznie długi okres obserwacji (tzw. horyzont obserwacji)
			\item wystarczające pobudzanie obiektu przez sygnały wejściowe (sygnały sterujące
					oraz zakłócenia mierzalne).
		\end{itemize}
		Problem obserwacji polega na wykorzystaniu danych wejściowych i wyjściowych
		obiektu do oszacowania wektora stanu układu w chwili początkowej
		przedziału czasu, w którym obserwacja/rejestracja była prowadzona.
		Symbolicznie można obserwator przedstawić jako blok:
					\begin{figure}[H]
			\centering
			\begin{tikzpicture}
				% Obswerwator
				\node [draw,
				fill=White, 
				minimum width=6cm, 
				minimum height=3cm,
				] (observer) {Obserwator};					
				% Arrows 

				\draw[-stealth] (observer.east) -- (5, 0 )
				node[very near end,below ]{$x(t-T)$};
				\draw[-stealth]  (-5, 0.81 ) -- (observer.165)
				node[very near start,below ]{$u(t)|^{t_k}_{t_0}$};
				\draw[-stealth]  (-5, -0.81 ) -- (observer.195)
				node[very near start,below ]{$x(t)|^{t_k}_{t_0}$};
				
			\end{tikzpicture}
			\caption{Blok obserwatora}
		\end{figure}

		Aby można było zrealizować takie zadanie, obiekt musi być obserwowalny.
	\section{Ogólna definicja obserwowalności}
		Układ o równaniach stanu i wyjścia postaci:
		\begin{figure}[H]
			\begin{align*}
				\dot{x}(t) = f(x(t), u(t),t) \\ 
				y(t) = h(x, u, t)
			\end{align*}
			\label{uklad}
		\end{figure}	
		jest obserwowalny, jeżeli przy dowolnym sterowaniu $u(t)$ w przedziale $<t0, t_0+ T)$, w
		chwili $t0+ T$ można wyznaczyć stan układu w chwili początkowej $x(t0)$, na podstawie
		zarejestrowanych sygnałów wyjściowych $y(t)|^{0+T}_t$
		oraz sygnałów wejściowych $u(t)|^k_{t_0}$ , gdzie $t_0$ oznacza czas 
		rzeczywisty rozpoczęcia rejestracji/obserwacji, a $T$ długość horyzontu obserwacji.

		Załóżmy, że $dim(x) = dim(f) = n$. O wektorowej, nieliniowej funkcji wyjścia $h(x , u, t)$
		załóżmy, że jest różniczkowalna względem wektora stanu $n$-krotnie, względem wektora
		sterowań $m$-krotnie i względem czasu jednokrotnie.

		Zdefiniujmy operator różniczkowy Lapunowa w następujący sposób:
		
		\begin{equation}
			L\{h\} \equiv \frac{\delta h}{\delta x} + \frac{\delta h}{\delta u} \dot{u} + \frac{\delta h}{\delta t}
		\end{equation}

		Zastosujmy $(n-1)$-krotnie operator Lapunowa do funkcji wyjść $h$.
		Otrzymamy równania definiujące fazowe zmienne stanu dla układu \ref{uklad}:

		\begin{align*}
			y = h(x,u,t) \\ 
			\dot{y} = L\{h(x,u,t)\} \\ 
			\ldots \ldots \ldots \ldots \\
			y^{(n-1)} = L^{(n-1)}\{h(x,u,t)\} \\ 
		\end{align*}
		Jest to układ $p$ równań algebraicznych i $p*(n-1)$ równań różniczkowych 1-go rzędu,
		definiujących $p$ podwektorów fazowych dla poszczególnych wyjść obiektu. Wymiary tych
		podwektorów odpowiadają najwyższym rzędom odpowiednich równań wejścia-wyjścia.
		Warunkiem koniecznym i wystarczającym istnienia rozwiązania takiego układu równań
		ze względu na $x$ , a tym samym lokalnej obserwowalności układu \eqref{uklad} jest spełnienie
		warunku:
		\begin{equation}
			rank  \left\{ \left(
			 \left. \frac{\delta h}{\delta x} \right)^T
			 \right\vert
			 \left( \frac{\delta }{\delta x} L\{h\} \right)^T 
			 \ldots 
			 \left( \frac{\delta }{\delta x} L^{(n-1)}\{h\} \right)^T
			 \right\} = n
		\end{equation}
		gdzie $n$ jest sumą wymiarów $p$ podwektorów fazowych poszczególnych wyjść. Macierz
		występująca w warunku jest macierzą Hankela.
		Dla modelu liniowego R. Kalman (1st IFAC Congress, Moskwa, 1960) podał prosty,
		algebraiczny warunek konieczny i wystarczający obserwowalności obiektu.
		W tym celu należy utworzyć tzw. macierz obserwowalności na podobnej zasadzie jak
		tworzona była macierz sterowalności:
		\begin{equation}
			O = [ C CA CA^2 \ldots CA^{n-1}]^T
		\end{equation}
		gdzie $n$ jest rzędem układu. Macierz $O$ składa się z $n$
		bloków $p$-kolumnowych, jej wymiary to $n \times (n \cdot p)$ .
		Warunkiem koniecznym i wystarczającym obserwowalności układu o modelu liniowym
		w przestrzeni stanu jest:
		\begin{equation}
			rank(O) = n
		\end{equation}
		Analogicznie jak w przypadku analizy sterowalności obiektu typu MIMO,
		przeprowadzonej na poprzednim wykładzie, można ustalić, czy w obiekcie występują
		zbędne z punktu widzenia pozyskiwania informacji o procesie, wyjścia. Można także
		sporządzić ranking wyjść, z punktu widzenia ich znaczenia dla efektywnego
		pozyskiwania informacji o stanie obiektu.

		Dany model układu można w ogólnym przypadku podzielić na część sterowalną i
		obserwowalną (S-O), sterowalną, ale nieobserwowalną (S-nO), obserwowalną, ale
		niesterowalną (nS-O) oraz niesterowalną i nieobserwowalną (nS-nO). Dla projektanta
		układu sterowania model układu musi być typu S-O. Jeżeli tak nie jest tp należy rozważyć
		uzupełnienie obiektu o dodatkowe urządzenia sterujące i sensory, by doprowadzić do
		sytuacji, że jest będzie on typu S-O.

		Dla obiektu obserwowalnego można zaprojektować obserwator. W przypadku
		analogowego modelu z czasem ciągłym obserwator może być zrealizowany sprzętowo,
		jako dedykowany układ przetwarzania sygnałów analogowych, albo po cyfryzacji modelu
		, jako algorytm przetwarzania próbek sygnałów wyjściowych i wejściowych w czasie rzeczywistym.

		\section{Liniowy obserwator analogowy (D. Luenberger, 1968)}

		Obserwator liniowy $r$-go rzędu liniowego obiektu $n$-go rzędu określają trzy macierze:
		\begin{equation}
			F \in M(r, r), H \in M(r, p), G \in M(r, m),
		\end{equation}
		gdzie symbol $M(r, r)$ oznacza macierz kwadratową o wymiarach $r \times r$, $M(r, p)$ macierz
		prostokątną o wymiarach, $r \times p$, $M(r, m)$ macierz prostokątną o wymiarach $r \times m$.
		$n$ – to wymiar wektora stanu obserwowanego obiektu, $p$ – liczba wyjść obserwowanego
		obiektu, $m$ – liczba jego wejść.
		Jeżeli $r = n$, to mamy przypadek obserwatora pełnego rzędu czyli takiego 
		szacującego wszystkie
		zmienne stanu obiektu na podstawie zapisów sygnałów wejściowych i wyjściowych.
		Gdy $r < n$, to mamy przypadek obserwatora zredukowanego szacującego tylko część zmiennych stanu.
		Równanie stanu obserwatora:
		\begin{equation}
			\dot{z}(t) = F z(t) + H y(t) + G u(t)
		\end{equation}
		gdzie $z$ - wektor stanu obserwatora, a rozwiązanie równania to $z(t)$ dla $(t > t_0)$ , gdzie $t_0$ to
		początek obserwacji.

		Macierze $F, H, G$ określają obserwator liniowy, jeżeli istnieją takie macierze $V \in M(n, p)$
		oraz $W \in M(n, r)$, że dla dowolnego sterowania $u(t)$ gdzie $(t> t_0)$ ,
		\begin{equation}
			x_E(t) = V y(t) + W z(t)
		\end{equation}
		jest estymatorem stanu obiektu i spełniony jest warunek:
		\begin{equation}
			\lim_{t \to \inf} \left. \left. \right\vert \right\vert 
			x(t) - x_E(t) 
			\left. \left. \right\vert \right\vert 
			\to
			0
			\label{warunek}
		\end{equation}
		Należy zauważyć, że oszacowanie $x_E (t)$ wektora stanu obiektu, zwane estymatą 
		stanu, wynika z wartości chwilowych wektora sygnałów
		wyjściowych obiektu i z chwilowych wartości wektora stanu obserwatora. Warunek \eqref{warunek}
		oznacza, że oczekujemy asymptotycznej zbieżności estymaty do wartości rzeczywistej
		przy nieskończonym horyzoncie obserwacji. Zbieżność estymaty wektora stanu do
		wartości rzeczywistej wektora obiektu sterowania z reguły ma charakter asymptotyczny
		$(T \to \inf)$ , pod warunkiem, że obserwator jest stabilny.
		W skończonym czasie jest możliwe jedynie zbliżenie się do wartości rzeczywistej:
		\begin{equation}
			\lim_{t \to t_0+T} \left. \left. \right\vert \right\vert 
			x(t) - x_E(t) 
			\left. \left. \right\vert \right\vert 
			< \varepsilon
		\end{equation}
	\section{Cecha wykrywalności}
		O parze macierzy $(A, C)$ z modelu \ref{uklad} mówimy, że posiada cechę wykrywalności, jeżeli
		istnieje taka macierz $L \in M(n, p)$, że macierz $A + LC$ jest stabilna.

		Warunkiem koniecznym wykrywalności jest obserwowalność obiektu, ale nie jest to
		warunek wystarczający. Bowiem z wykrywalności wynika obserwowalność, ale z
		obserwowalności nie wynika wykrywalność, gdyż dla konkretnych $A, C$ może nie istnieć
		$L$ spełniająca warunek z definicji. Oznacza to, że w takim przypadku nie można
		skonstruować stabilnego obserwatora.

		Jeżeli jednak warunek ten jest spełniony, to obserwator pełnego rzędu $(r=n)$ jest
		określony następującymi równaniami stanu i wyjścia:
		\begin{align*}
			\dot{z}(t) = [A + LC]z(t) - Ly(t) + Bu(t) \\ 
			x_e(t) = z(t)
		\end{align*}
		W takim przypadku
		\begin{equation}
			F = [A+LC], H = -L , G=B, V=0, W=I
		\end{equation}
		Czyli oszacowaniem wartości wektora stanu obiektu w chwili początkowej $t_0$
		jest chwilowa wartość wektora stanu obserwatora pełnego rzędu $z(T)$
		dostępna po upływie czasu $T$.Zakładamy przy tym zerowe warunki początkowe obserwatora.
		Tymczasem rzeczywisty stan obiektu, istotny z punktu widzenia sterowania jest już wówczas równy:
		\begin{equation}
			x(t_0+T) = e^{A(t_0+T)}x(t_0)+\int^{t_0+T}_{t_0}e^{A(t0+T-\tau)}Bu(\tau)d\tau
		\end{equation}
		Łatwo jednak zaktualizować (odnieść do bieżącej chwili) czyli obliczyć "in no time"
		bieżącą estymatę wektora stanu posługując się wzorem:
		\begin{equation}
			x_E(t) = e^{A(t)} x_E(t_0)+\int^t_{t_0}e^{A(t-\tau)} Bu(\tau)d\tau
		\end{equation}
		przy czym $t > t_0 + T$. $A, B$ są macierzami modelu w przestrzeni stanu obiektu. Jest to
		aktualizacja estymaty oparta o znany model obiektu. Istotne jest jednak, by moc
		obliczeniowa komputera sterującego pozwalała zrealizować obliczenia/symulację w
		bardzo krótkim czasie w porównaniu z dynamiką rzeczywistego obiektu.

		Fazę obserwacji bądź też zbierania danych można skrócić jeśli dopuścimy
		odchyłkę estymaty od wartości rzeczywistego stanu obiektu.
		Realizujemy w trakcie zbierania danych specjalnie zaprojektowany program 
		stymulacji obiektu testowymi sygnałami wejściowymi, czyli przeprowadzając 
		tzw. czynny eksperyment identyfikacji.
		Warto zauważyć, że gdy obiekt jest stabilny, to błąd estymacji $e(t)=x(t)-x_E(t)$ maleje
		asymptotyczne do zera ze wzrostem $t$:
		\begin{equation}
			\lim_{t \to \inf} [ x(t) x_E (t)] = \lim_{t \to \inf} e^{At}[ x(t_0) x_E (t_0)] = 0 
		\end{equation}
		Aktualizacji na bieżąco dokonuje obserwator. Dostarcza on na bieżąco oszacowanie
		wartości wektora stanu w chwili wcześniejszej od chwili bieżącej o $T$.
		Warto porównać ją z wartościami
		generowanymi na modelu symulacyjnym, po przyjęciu jako stan początkowy $x_E (t_0)$ –
		wyniku pierwszej (może być, że jednorazowej) pracy obserwatora w oparciu o zapisy
		sygnałów w przedziale czasowym $[t_0 , t_0+ T)$.
\end{document}
