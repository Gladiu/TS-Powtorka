\author{Kasiński}
\title{Wykład 12}
\documentclass{article}
\usepackage[margin=1.4cm]{geometry}
\usepackage[utf8]{inputenc}
\usepackage{babel}
\usepackage{polski}
\usepackage{float}
\usepackage{graphicx}
\usepackage{amsmath}
\usepackage{hyperref}
% More defined colors
\usepackage[dvipsnames]{xcolor}
% Required package
\usepackage{tikz}
\usetikzlibrary{positioning}
\begin{document}
	\maketitle
	\section{Stabilność nieliniowych obiektów sterowania}
		Stabilny układ dynamiczny, to taki układ, który startując z niezerowego stanu początkowego
		, pozostaje w tym otoczeniu. Zamiast punktu pracy może to być trajektoria
		nominalna zdefiniowana w przestrzeni stanu. Wówczas, za stabilny uznamy układ, który
		podąża wskutek odpowiedniego sterowania wzdłuż tej trajektorii, a jego stan pozostaje w
		każdej chwili w niewielkim otoczeniu trajektorii nominalnej, a trajektoria rzeczywista nie oddala
		się od nominalnej w każdej chwili t więcej niż o $\varepsilon$.
		Układ równań stanu nieliniowego/niestacjonarnego obiektu sterowania;
		\begin{equation}
			\dot{x}(t) = f(x(t), u(t), t)
			\label{nieliniowy-niestacjonarny-obiekt-sterowania}
		\end{equation}
		Układ ten jest $n$-tego rzędu, o $m$ sygnałach sterujących ($dim(x) = n, dim(u) = m$). Jest to układ
		nieautonomiczny. Załóżmy, ze jest on sterowany z zastosowaniem nieliniowych i
		niestacjonarnych praw sterowania typu sprzężenie zwrotne od stanu:
		\begin{equation}
			u = g(x,t)
		\end{equation}
		gdzie $m$-wymiarowa, wektorowa, nieliniowa, jawnie zależna od czasu funkcja $g$ jest funkcją
		sprzężeń zwrotnych. Jawna zależność od czasu
		prawa sterowania odpowiada realizacji pewnego programu sterowania, zmiennego w czasie
		rzeczywistym, bezpośrednio po wytrąceniu obiektu ze stanu równowagi i występuje np. w
		układach sterowania adaptacyjnego. Po podstawieniu jej do równania \eqref{nieliniowy-niestacjonarny-obiekt-sterowania}
		dokonujemy tzw. autonomizacji modelu obiektu:
		\begin{equation}
			\dot{x}(t) = f(x(t), g(x(t),t),t)
		\end{equation}
		by po przekształceniach (ponownych pogrupowaniach zmiennych stanu) otrzymać model
		dynamiki układu zamkniętego:
		\begin{equation}
			\dot{x}(t) = f^*(x(t), t)
			\label{model-dynamiki-ukladu-zamknietego}
		\end{equation}
		gdzie $f^*$ jest "przebudowaną" funkcją momentum. $f$ zostaje zmodyfikowana przez
		wprowadzenie funkcji $g$ w miejsce $u$, możemy stwierdzić, żę $f$ jest superpozycją nad $g$.
		\eqref{model-dynamiki-ukladu-zamknietego} jest modelem dynamiki nieliniowego układu zamkniętego.

		Rozwiązaniem układu $n$ równań różniczkowych 1-go rzędu jest (w przypadku nie
		zdegenerowanym) właściwa, $n$-wymiarowa krzywa parametryczna $x(t)$ w $n$-wymiarowej
		rozmaitości różniczkowej stanu zwana trajektorią stanu. Parametrem tej krzywej jest czas $t$.
		
		\begin{verbatim}
			Nieliniowy układ dynamiczny opisany równaniem \eqref{model-dynamiki-ukladu-zamknietego}
			nazywamy autonomicznym, jeżeli funkcja $f^*$ nie zależy jawnie od $t$.
		\end{verbatim}

		W zasadzie wszystkie układy fizyczne nie są autonomiczne (gdyż się starzeją, zużywają), ale
		jeżeli zmiany ich właściwości są wolnozmienne w zestawieniu z dynamiką zmian ich stanu, to
		możemy je traktować jako autonomiczne w określonych przedziałach czasowych.
		Nawet obiekt, którego wystarczająco dokładnym modelem dynamiki jest model liniowy w
		przestrzeni stanu, po zastosowaniu nieliniowych reguł sterowania opartych np. o zasadę
		sprzężeń zwrotnych, staje się nieliniowym systemem dynamicznym. A jeżeli reguła
		sterowania zawiera czynnik czasu, to dodatkowo staje się systemem nieautonomicznym.
		\begin{verbatim}
			OBIEKT LINIOWY + REGULATOR LINIOWY = LINIOWY SYSTEM DYNAMICZNY
		\end{verbatim}

		Obiekt sterowania o macierzach ($A$, $B$) modelu w przestrzeni stanu oraz regulator. Daje w rezultacie
		System autonomiczny, o zmodyfikowanej macierzy stanu $[A+KB]$ – o przesuniętych biegunach
		systemu w następstwie odpowiedniego doboru elementów macierzy $K$. Macierz ta zawiera współczynniki
		wzmocnień odpowiednich pętli ujemnych sprzężeń zwrotnych, stabilizujących obiekt
		niestabilny lub poprawiających jego zapasy stabilności.
		\begin{verbatim}
			OBIEKT LINIOWY + REGULATOR NIELINIOWY = NIELINIOWY SYSTEM DYNAMICZNY
			OBIEKT LINIOWY + REGULATOR PROGRAMOWANY = SYSTEM NIEAUTONOMICZNY
		\end{verbatim}
		Trajektorię stanu może stanowić pojedynczy punkt w przestrzeni stanu zwany jest wówczas
		punktem równowagi trwałej.
		\begin{verbatim}
			Def. Stan $x_r$ nazywamy stanem równowagi trwałej, jeżeli z tego że dla pewnego $t$ $x(t) = x_r$
			wynika, że dla dowolnego $\tau > t x(\tau) = x_r$.
		\end{verbatim}
		Tym samym mamy, że $f(x_r) = 0$, czyli że punkt równowagi trwałej jest punktem stacjonarnym
		układu autonomicznego.
		Istnieją jeszcze inne punkty stacjonarne poza punktami równowagi trwałej np. punkty
		równowagi nietrwałej/chwiejnej.
		Model liniowy ma tylko jeden punkt równowagi trwałej: $x_r = 0$ gdy macierz układu $A$ jest
		pełnego rzędu. W przeciwnym wypadku ma on nieskończenie wiele
		punktów równowagi, czyli gdy defekt rzędu jest równy jeden: $rank(A) = n-1$, to miejscem
		geometrycznym punktów równowagi jest prosta w przestrzeni stanu; a gdy $rank(A) = n-2$, jest
		to pewna płaszczyzna w $n$-wymiarowej przestrzeni stanu, czyli punkty równowagi nie są
		wówczas izolowane – nie są zbiorami miary zero. Ogólnie można ten warunek zapisać, że
		\begin{equation}
			x_r \in Ker(A)
		\end{equation}
		W przypadku modeli nieliniowych rzecz przedstawia się odmiennie. Mogą one mieć więcej niż
		jeden izolowany punkt stacjonarny, w tym więcej niż jeden punkt równowagi trwałej,
		Takie punkty zwane są atraktorami i są lokalnymi centrami tzw. obszarów przyciągania
		bądź też basenów atrakcji. Ilość punktów równowagi nie wiąże się z utratą rzędu
		rozmaitości stanu, nawet tej nieliniowej nieliniowej przestrzeni stanu.
		W przypadku utraty pełnej wymiarowości przez rozmaitość stanu, czyli nastąpienia tzw. defektu rzędu, punkty
		stacjonarne tworzą podrozmaitości osobliwe,
		o wymiarowości zależnej od stopnia defektu rzędu rozmaitości stanu. W takim
		przypadku mówimy o ko-wymiarze podrozmaitości osobliwej. Szczególnym przypadkiem są
		np. krzywe zamknięte jako zbiory stacjonarne tzw. orbity np. kołowe, eliptyczne.
		Jeżeli w konkretnym przypadku mamy kilka izolowanych punktów stacjonarnych, to
		każdorazowo dokonuje się równoległego przeniesienia układu współrzędnych stanu
		tak, aby jego początek pokrywał się z wybranym do analizy punktem stacjonarnym.
		Mówimy wówczas o lokalnym, związanym z danym punktem układzie współrzędnych stanu.
		Redefiniujemy wówczas zmienne stanu w następujący sposób:
		\begin{equation}
			\xi(t) = x(t)-x_r \to x(t) = \xi(t) + x_r \to \xi = f^*(\xi+x_r),	
		\end{equation}
	\section{Ruch, trajektoria nominalna, stabilność ruchu względem trajektorii nominalnej}
		\subsection{Pytanie}
			Czy układ dynamiczny utrzyma się na swojej trajektorii nominalnej (trajektorii zadanej, którą
			powinien odtwarzać) jeżeli jego stan zostanie poddany lekkiemu zaburzeniu (perturbacji)?
			Na to pytanie można odpowiedzieć poprzez redukcję powyższego problemu do badania
			stabilności w otoczeniu punktu równowagi równoważnego układu nieautonomicznego.

			Niech $x^*(t)$ oznacza rozwiązanie równania $\dot{x}(t) = f^*(x(t),t)$, przy warunku
			początkowym $x^*(0) = x_0$ równym warunkowi początkowemu trajektorii nominalnej,

			Niech $x^*(0) = x_0 + \Delta x$, gdzie $\Delta x$ oznacza zaburzenie stanu początkowego.
			Wówczas $x(t)$ - rozwiązanie równania dynamiki $ \dot{x}(t) = f^*(x(t),t)$ dla zaburzonego
			stanu początkowego jest trajektorią zaburzoną. Zdefiniujmy dewiację/odchyłkę
			trajektorii zaburzonej względem trajektorii nominalnej jako:
			\begin{equation}
				e(t) = x(t) – x^*(t)
			\end{equation}
			Równanie dynamiki odchyłki ma postać:
			\begin{equation}
				\dot{e}(t) = f^*(x^*(t)+e(t),t) - f^*(x^*(t),t) = h(e(t),t)
			\end{equation}
			gdzie $e(0) = \Delta x$
			Jeżeli $h(0,t) = 0$, to wprowadzony (uchybowy) układ dynamiczny ma punkt równowagi
			w początku przeniesionego układu współrzędnych stanu.

			Dynamika perturbacyjna $\dot{e}(t) = h(e(t),t)$ jest nieautonomiczna ze względu na obecność
			w funkcji $h$ trajektorii nominalnej $x^*(t)$ będącej jawną funkcją czasu, którą można
			interpretować jako swego rodzaju wymuszenie zewnętrzne dynamiki perturbacyjnej.
		\subsection{Przykład}
			Dany jest nieliniowy układ autonomiczny o równaniu:
			\begin{equation}
				m\ddot{x} + k_1x + k_2 x_3 = 0
			\end{equation}
			Równanie opisuje ruch o jednym stopniu swobody środka masy $m$, osadzonej na twardej
			sprężynie z pomijalnym tłumieniem.
			Jak wygląda trajektoria nominalna $x^*(t)$, przy warunku początkowym $x^*(0) = x_0$ i jak
			wygląda trajektoria $x(t)$, po zaburzeniu warunku początkowego $\Delta x$ ?
			Równanie dynamiki perturbacyjnej ma postać:
			\begin{equation}
				m\ddot{e} + k_1e + k_2\{e_3 + 3e^2x*(t) + 3e [x^*(t)]2 \}= 0.
			\end{equation}
			Jak widać na tym przykładzie, dynamika perturbacyjna nie jest autonomiczna bo
			równania stanu tego układu $x_1 = e, x_2 = \dot{e} $ jawnie zależą od czasu (poprzez obecność
			w drugim równaniu $x^*(t)$).
			Natomiast w przypadku autonomicznego modelu liniowego, równanie dynamiki
			perturbacyjnej jest nadal autonomiczne: $\dot{e}(t) = A e(t)$.
			W dynamice nieliniowej rozróżniamy następujące kategorie stabilności:
			\begin{itemize}
				\item stabilność asymptotyczną,
				\item stabilność wykładniczą (z wykładniczą zależnością kryterium od czasu),
				\item stabilność globalną (absolutną).
			\end{itemize}
			Niech $ | x | < R$ oznacza wnętrze hiperkuli o promieniu $R$ i o środku w wybranym $i$-tym
			punkcie stacjonarnym $x_s^i$, w $n$-wymiarowej, metrycznej przestrzeni stanu.
			Oznaczana jest $B_n(x_s^i, R)$. Układ współrzędnych stanu został równolegle przeniesiony do
			punktu stacjonarnego w taki sposób, że stanowi on teraz początek przeniesionego
			układu współrzędnych. Po translacji układu współrzędnych hiperkulę oznaczamy
			$B_n(0,R)$ lub krótko $B(R)$.
			\begin{verbatim}
				Stan równowagi $x_s^i$ jest stabilny (w nowym, przeniesionym układzie współrzędnych jest
				to x =0), jeżeli dla dowolnego R >0 istnieje r>0 takie, że:
				||x(0) ||< r \to ||x(t)||< R
				$\forall t > 0$.
				Jest to definicja stabilności w sensie Lapunowa. Wyrażona w stylu Cauchy’ego ($\forall R \foreach r).
			\end{verbatim}
\end{document}
