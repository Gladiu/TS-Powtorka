\author{Kasiński}
\title{Wykład 9}
\documentclass{article}
\usepackage[margin=1.4cm]{geometry}
\usepackage[utf8]{inputenc}
\usepackage{babel}
\usepackage{polski}
\usepackage{float}
\usepackage{graphicx}
\usepackage{amsmath}
\usepackage{hyperref}
\begin{document}
	\maketitle
	\section{Badanie obserwowalności obiektu}
		Dysponując adekwatnym modelem analitycznym obiektu sterowania można w oparciu
		o ten model zbadać jego obserwowalność.
		Zakładając liniowość i stacjonarność modelu obiektu sterowania mamy:
		\begin{align*}
			\dot{x} = Ax(t)+Bu(t) \\
			y = Cx(t)+Cu(t) 
		\end{align*}
		W przypadku sterowalności istotne było wykazanie zależności kompletnej trajektorii
		wektora stanu (krzywej właściwej, czyli pełnowymiarowej, przebiegającej w n wymia-
		rowej przestrzeni stanu) od sygnałów sterujących działających w pewnym przedziale
		czasu.
\end{document}
