\author{Kasiński}
\title{Wykład 8}
\documentclass{article}
\usepackage[margin=1.4cm]{geometry}
\usepackage[utf8]{inputenc}
\usepackage{babel}
\usepackage{polski}
\usepackage{float}
\usepackage{graphicx}
\usepackage{amsmath}
\usepackage{hyperref}
\begin{document}
	\maketitle
	\section{Badanie Sterowalności obiektu}
		Dysponując adekwatnym modelem analitycznym obiektu sterowania można zbadać jego
		sterowalność w oparciu o ten model. Ogólna postać stanowego modelu (równania stanu i
		równania wyjścia), przy założeniu, że jest to obiekt o skupionych parametrach , jest
		następująca:
		\begin{align*}
			\dot{x(t)} = f(x(t), u(t), t) \\
			y(t) = h(x,u,t)
		\end{align*}
		pole wektorowe f jest wektorem funkcji (w ogólnym przypadku nieliniowych), których
		argumentami są wybrane zmienne stanu i sygnały sterujące, mające wpływ na daną składową
		wektora prędkości zmian stanu.

		Niestacjonarność właściwości dynamicznych i statycznych obiektu zaznacza się, poprzez
		jawną obecność czasu jako argumentu w równaniach modelu.

		Jeżeli w czasie pracy obiektu ulega zmianie jego struktura powiązań wejścia-wyjścia, to
		mówimy o obiekcie ze zmienną strukturą. Po każdym takim incydencie powinna być
		przeprowadzona adekwatna zmiana w strukturze modelu.

		Analiza zachowania obiektu o zmiennej strukturze wymaga posługiwania się wieloma
		modelami przełączanymi w chwilach zmiany struktury i traktowania wartości końcowych
		zmiennych występujących w poprzedniej strukturze jako warunki początkowe w
		strukturze aktualnej.

		Przykładem obiektu sterowania o zmiennej w trakcie misji strukturze równań stanu i wyjścia
		jest samolot o zmiennej geometrii płatów nośnych, rakieta wielostopniowa, robot-
		manipulator wykonujący zdanie wymagające interakcji z otoczeniem 

		Niestacjonarność może cechować obiekt o stałej strukturze, którego właściwości ulegają
		zmianie w czasie, w funkcji przebiegu (kilometrażu) wskutek procesów starzenia, zużycia, lub
		przekroczenia gwarancji (np. granicznej liczby cykli roboczych). W tym przypadku jest ona
		modelowana poprzez jawne uzależnienie od czasu wybranych parametrów modelu
		eksploatacyjnego obiektu. W odróżnieniu od niestacjonarności strukturalnej tego rodzaju
		niestacjonarność określamy mianem niestacjonarności parametrycznej.

		Zazwyczaj dynamika zmian wartości takich parametrów jest duże wolniejsza od dynamiki
		własnej obiektu, co pozwala analizować go jako obiekt odcinkami stacjonarny, z aktualizacją
		wartości wybranych parametrów modelu (odnowa modelu).

		Jeżeli procesy zużycia mają dużą dynamikę (przypadek silnika rakietowego), zbliżoną do
		dynamiki procesu, to niestacjonarność parametryczna musi być jawnie ujęta w równaniu
		stanu.
		\begin{verbatim}
			Nie istnieje w ogólnym przypadku bezpośrednia metoda
			badania sterowalności obiektu na podstawie jego nieliniowego
			modelu w przestrzeni stanu!
		\end{verbatim}

		Natomiast, na podstawie liniowych równań stanu (modelu lokalnie przybliżającego dynamikę
		nieliniową) można przeprowadzić test sterowalności opierając się o twierdzenie R. Kalmana.
		W przypadku liniowym, model niestacjonarny przyjmuje postać:
		\begin{align*}
			\dot{x(t)} = A(t)x(t) + B(t)u(t) \\
			y(t) = C(t)x(t) + D(t)u(t)
		\end{align*}
		gdzie niestacjonarność „wbudowano” w parametry modelu.
		Zakładając stacjonarność obiektu mamy:
		\begin{align*}
			\dot{x(t)} = Ax(t) + Bu(t) \\
			y(t) = Cx(t) + Du(t)
		\end{align*}
		Macierz stanu $A$ reprezentuje model autonomiczny $ u(t) = 0 $, a trajektoria stanu
		modelu autonomicznego jest obrazem ruchu swobodnego.

		Kolumny macierzy sterowań $B = [b_1 b_2 \cdots b_m]$ należy zinterpretować jako kierunkowe
		współczynniki wzmocnienia skalarnych sterowań odpowiednio: $u_1 u_2 \cdots u_m$ . Decydują
		one o efekcie kierunkowym sterowania na określonym wejściu. Natomiast o wielkości
		wpływu danego sterowania na prędkości zmian wektora stanu decyduje amplituda
		(chwilowa wartość) danego sterowania. Jeżeli amplituda jest ograniczona 
		to wpływ ten jest również ograniczony lub skwantowany.

		Sterowalność obiektu oznacza taką jego właściwość,
		że jeżeli znajduje się on w
		dowolnym stanie początkowym, to przez synergię ruchu własnego (swobodnego) i
		odpowiednie sterowanie może osiągnąć dowolny stan końcowy (zadany).

		Przy czym nie stawiamy dodatkowych warunków np. ograniczających czas osiągnięcia
		stanu końcowego, czy ograniczających wartości chwilowe sygnałów sterujących.

		Tak rozumiana sterowalność wynika jedynie z geometrycznych konsekwencji
		zastosowania danego modelu obiektu w przestrzeni stanu. Mnożenie wektora stanu
		przez macierz $A$ można rozpatrywać jako swego rodzaju 
		rzutowanie $X^n \rightarrow TX^n$ (elementu
		wektorowej przestrzeni stanu Xn w element hiperpłaszczyzny $TX^n$,
		stycznej do $X^n$ ), czyli złożenie operacji
		obrotu wektora w przestrzeni n-wymiarowej i zmiany skali
		(współczynnikiem zmiany skali wektora jest wartość $det A$). Natomiast mnożenie
		wektora sterowań przez macierz sterowań można rozpatrywać jako swego rodzaju
		rzutowanie $U^m \rightarrow TX^n$ (elementu wektorowej przestrzeni sterowań
		$U^m$ w element
		hiperpłaszczyzny $TX^n$), a jest to w równaniu stanu składnik addytywny. Macierz $AB$
		reprezentuje operator rzutowania efektu sterowania w hiperpłaszczyznę prędkości $TX^n$.
		Utwórzmy następującą macierz:
		\begin{equation}
			S = [ B AB A^2B \cdots A^{n-1}B]
		\end{equation}
		Gdzie $n$ jest liczbą wymiarów modelu.
		Macierz $S$ nosi nazwę macierzy sterowalności i znajduje zastosowanie w
		badaniu sterowalności modelu liniowego.

		\begin{verbatim}
			Na to, by obiekt liniowy o równaniach był sterowalny potrzeba i wystarcza, że
			spełniona jest równość
		\end{verbatim}
		\begin{equation}
			rank(s) = n
		\end{equation}
		Macierz sterowalności ma wymiary $n\times(n\times m)$ 
		i składa się z $n$ bloków $m$-kolumnowych.
		Kolejny blok uzyskujemy rekurencyjnie z poprzedniego poprzez przemnożenie go przez
		$A$. Opłaca się badać, czy warunek rzędu jest spełniony po uzupełnieniu macierzy o kolejny
		$i$-ty blok (dla niekompletnej macierzy $S_i$).
		Jeżeli jest spełniony, to znaczy, że obiekt jest
		sterowalny i dalsze obliczenia kolejnych bloków są niecelowe. Niska wartość wskaźnika
		$i$ świadczy o malej roli ruchu swobodnego w zapewnieniu obiektowi sterowalności i o
		dominującej roli struktury wielowymiarowego układu sterowania (stopnia powiązań
		sygnałów wejściowych z sygnałami zmiennych stanu).

		Warto zastanowić się nad rolą poszczególnych sygnałów sterujących w zapewnieniu
		sterowalności obiektu i zapewnieniu efektywnego sterowania. W wielowymiarowym
		obiekcie sterowania nie wszystkie sygnały sterujące są jednakowo ważne z punktu
		widzenia sterowalności obiektu oraz efektywności sterowania.
		\subsection{Przykład}
			Dany jest wielowymiarowy obiekt liniowy $n$-tego rzędu 
			o czterech wejściach sterujących.
			Przeprowadźmy negatywną analizę jego sterowalności 
			z punktu widzenia poszczególnych wejść.
			\begin{itemize}
				\item Pomińmy i-ty sygnał sterujący (i= 1,2,3,4).
					Mamy 4 trójelementowe kombinacje: 
					$(u_1 , u_2 , u_3), (u_1 , u_2 , u_4), (u_1 , u_3 , u_4) ,
					(u_2 , u_3 , u_4)$. Każda odpowiada usunięciu z macierzy
					sterowań B odpowiednio 4-tej, 3-ciej, 2-giej, 1-wszej kolumny.
				\item Przeprowadźmy dla każdej z kombinacji test sterowalności.
					Jeżeli dla którejś z nich
					wynik testu jest negatywny, to oznacza to,
					że usunięte wejście jest niezbędne dla
					zapewnienia sterowalności układu.
					Takie wejście nazywamy krytycznym.
				\item Jeżeli dla wybranej konfiguracji wejść wynik jest pozytywny,
					to oznacza to, że dane
					usunięte wejście nie jest niezbędne dla zapewnienia
					sterowalności obiektu, a kombinacje
					dla których test jest pozytywny można uznać
					za alternatywne (antyawaryjne).
				\item W takim przypadku należy sprawdzić jak zmieniła
					się wartość indeksu $j$ macierzy
					sterowalności obiektu zredukowanego $S_j$.
					Gdzie $S_j$ to
					minimalna wartość wykładnika potęgi
					macierzy $A$ w tej macierzy, przy której częściowa 
					macierz sterowalności $S_j$ spełnia
					warunek rzędu. Zawsze spełniony jest warunek $i\leq j$.
					Jeżeli $i=j$, to konfiguracja 3-
					wejsciowa jest ekwiwalentna nadmiarowej konfiguracji 
					4-wejściowej. Jeżeli $j > i$,
					to konfigurację 3-wejściową można uznać za
					zapasową. Stopień odporności 
					obiektu na awarie 
					rośnie wraz z liczbą zredukowanych kombinacji zapasowych.
			\end{itemize}
			Powyższe rozważania łatwo uogólnić na przypadek układu z $m$ wejściami. Wówczas
			operacje usuwania wejść można rozszerzyć na par wejść, trójki wejść, itd.

			Innym aspektem analizy właściwości sterowniczych obiektu wielowymiarowego jest
			uwzględnienie ograniczeń wartości chwilowych sygnałów sterujących oraz zbadanie
			relacji pomiędzy celem sterowania a dynamiką własną obiektu. Sprawę można prościej
			przedyskutować w przypadku obiektów sterowanych dyskretnie w czasie.

			Celem sterowania jest zazwyczaj przeprowadzenie obiektu z pewnego stanu
			początkowego do zadanego stanu końcowego. W przestrzeni stanu cel reprezentuje
			wektor $x_N – x0$. Cel ten posiada określony kierunek, zwrot i długość).
			Sterowanie chwilowo
			sprzyjające realizacji celu charakteryzuje się dodatnim iloczynem skalarnym
			\begin{equation}
				<x_N-x_0, Bu_k> > 0
			\end{equation}
			Predykcja wartości wektora stanu w chwili następnej na podstawie aktualnej wartości
			wektora stanu i podjętej, aktualnej decyzji sterującej wynika z równania:
			\begin{equation}
				x_{k+1} = A^*x_k + B^*u_k
			\end{equation}
			Gdzie $A^*$ oraz $B^*$ są odpowiednio macierzą stanu i macierzą sterowań w modelu dyskretnym.
			\begin{align*}
				x_N = A^*x_{N-1} + B^*u_{N-1} \\
				x_N = A^*[A^*x_{N-2}+B^*u_{N-2}] + B^*u_{N-1} \\
				x_N = (A^*)^Nx_0+\sum^{N-1}_{k=0}(A^*)^{N-1-k}B^*u_k
			\end{align*}
			Przy danych $x_0$ oraz $x_N$ należy rozwiązać to równanie 
			ze względu na sekwencję sterowań
			\begin{equation}
				u_0,u_1,u_2,\cdots u_{N-1}
			\end{equation}

			Jest to układ $n$ równań liniowych z $(N-1) \times m$ niewiadomymi.
			Aby był określony macierz układu (macierz blokowa złożona z bloków odpowiadających
			wyrażeniom pod znakiem $\sum$) musi być rzędu $n$.
			Wówczas jedynie gdy $(N-1) \times m =n$
			uzyskujemy jednoznaczne rozwiązanie, tj. sekwencję sterującą realizującą cel.
			Można sprawdzić jak wiele decyzji sterujących potrzebne jest do tego. W tym celu
			poszukujemy $N$ spełniającego powyższą zależność. Takie działanie ma sens, gdy liczba
			zmiennych stanu jest całkowitą wielokrotnością liczby wejść.
			W takim przypadku wyznaczone jednoznacznie rozwiązanie powyższego
			równania określa najkrótszą sekwencję sterującą
			przeprowadzających układ z danego stanu początkowego do
			zadanego stanu końcowego.
			Gdy $m=n$ jest to sekwencja jednoelementowa. Występuje w przypadku
			układu o dynamice kanałów
			wejścia-wyjścia 1-go rzędu i o odpowiednio dużej liczbie wejść. W przypadku kanałów o
			dynamice wyższego rzędu, długość sekwencji sterującej nie może być mniejsza niż
			maksymalny rząd kanału.
\end{document}
