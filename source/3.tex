\documentclass{article}
\usepackage[utf8]{inputenc}
\usepackage{babel}
\usepackage{polski}
\usepackage{float}
\title{Wykład 3}
\maketitle
\begin{document}
	\section{Podejście strukturalne do wyprowadzania modelu stanowego układu złożonego}

		Złożony obiekt sterowania powstaje w wyniku połączenia wielu urządzeń wykonawczych/obróbczych lub
		innych mniejszych obiektów  w sieć przepływu sygnałów charakteryzujących wzajemne
		oddziaływania między tymi urządzeniami.
		Same połączone urządzenia, które na schemacie funkcjonalnym są
		symbolicznie przedstawione jako bloki $U_i$ mogą być same w sobie złożone ze względu na
		dynamikę własną i właściwości statyczne. Wówczas należy je poddać dekompozycji na
		elementarne człony typu SISO połączone wewnętrznie ze sobą.
		
		Za człony elementarne uważać będziemy takie części analizowanego urządzenia, których
		dynamikę można opisać równaniami wejścia-wyjścia niskiego rzędu,
		wyprowadzonymi z klasycznych praw fizyki takich jak: prawa Newtona, prawa Ohma,
		Kirchoffa, prawa Bernouillego. Prawa te odnoszą się do obiektów
		takich, których wymiary liniowe są znacznie mniejsze niż długości fal
		przechodzących przez nie sygnałów. O takich obiektach mówimy, że są obiektami
		o parametrach skupionych.

		Rozróżniamy grupę elementów o parametrach rozłożonych takie jak:
		\begin{itemize}
			-item Urządzenia cieplne 
			-item Reaktory chemiczne dużych gabarytów
			-item Długie rurociągi
			-item Długie linie energetyczne
			-item Obwody mikrofalowe
			-item Plastyczne struktury mechaniczne
		\end{itemize}

		Elementy te charakteryzują się dodatkowym etapem analizy dynamicznej
		w sytuacji, gdy nie można jawnie rozwiązać opisujących ich dynamikę równań cząstkowych.
		Tym etapem jest jest dyskretyzacja przestrzenna równań cząstkowych czyli dekompozycja obiektu na układ równań
		zwyczajnych powiązanych poprzez warunki, które są stałe i niezmienne w odpowiednich punktach
		oraz muszą być spełnione, aby uzyskać rozwiązanie. Takie warunki nazywamy warunkami brzegowymi.

		Podsumowując, przy elementach o parametrach rozłożonych konieczna jest dyskretyzacja przestrzenna równań
		cząstkowych. Dyskretyzacja przestrzenna równań cząstkowych to dekompozycja obiektu na układ równań zwyczajnych
		powiązanych przez warunki brzegowe.
\end{document}
