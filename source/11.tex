\author{Kasiński}
\title{Wykład 12}
\documentclass{article}
\usepackage[margin=1.4cm]{geometry}
\usepackage[utf8]{inputenc}
\usepackage{babel}
\usepackage{polski}
\usepackage{float}
\usepackage{graphicx}
\usepackage{amsmath}
\usepackage{hyperref}
% More defined colors
\usepackage[dvipsnames]{xcolor}
% Required package
\usepackage{tikz}
\usetikzlibrary{positioning}
\begin{document}
	\maketitle
	\section{Stabilność nieliniowych obiektów sterowania}
		Stabilny układ dynamiczny, to taki układ, który startując z niezerowego stanu początkowego
		, pozostaje w tym otoczeniu. Zamiast punktu pracy może to być trajektoria
		nominalna zdefiniowana w przestrzeni stanu. Wówczas, za stabilny uznamy układ, który
		podąża wskutek odpowiedniego sterowania wzdłuż tej trajektorii, a jego stan pozostaje w
		każdej chwili w niewielkim otoczeniu trajektorii nominalnej, a trajektoria rzeczywista nie oddala
		się od nominalnej w każdej chwili t więcej niż o $\varepsilon$.
		Układ równań stanu nieliniowego/niestacjonarnego obiektu sterowania;
		\begin{equation}
			\dot{x}(t) = f(x(t), u(t), t)
			\label{nieliniowy-niestacjonarny-obiekt-sterowania}
		\end{equation}
		Układ ten jest $n$-tego rzędu, o $m$ sygnałach sterujących ($dim(x) = n, dim(u) = m$). Jest to układ
		nieautonomiczny. Załóżmy, ze jest on sterowany z zastosowaniem nieliniowych i
		niestacjonarnych praw sterowania typu sprzężenie zwrotne od stanu:
		\begin{equation}
			u = g(x,t)
		\end{equation}
		gdzie $m$-wymiarowa, wektorowa, nieliniowa, jawnie zależna od czasu funkcja $g$ jest funkcją
		sprzężeń zwrotnych. Jawna zależność od czasu
		prawa sterowania odpowiada realizacji pewnego programu sterowania, zmiennego w czasie
		rzeczywistym, bezpośrednio po wytrąceniu obiektu ze stanu równowagi i występuje np. w
		układach sterowania adaptacyjnego. Po podstawieniu jej do równania \eqref{nieliniowy-niestacjonarny-obiekt-sterowania}
		dokonujemy tzw. autonomizacji modelu obiektu:
		\begin{equation}
			\dot{x}(t) = f(x(t), g(x(t),t),t)
		\end{equation}
		by po przekształceniach (ponownych pogrupowaniach zmiennych stanu) otrzymać model
		dynamiki układu zamkniętego:
		\begin{equation}
			\dot{x}(t) = f^*(x(t), t)
			\label{model-dynamiki-ukladu-zamknietego}
		\end{equation}
		gdzie $f^*$ jest "przebudowaną" funkcją momentum. $f$ zostaje zmodyfikowana przez
		wprowadzenie funkcji $g$ w miejsce $u$, możemy stwierdzić, żę $f$ jest superpozycją nad $g$.
		\eqref{model-dynamiki-ukladu-zamknietego} jest modelem dynamiki nieliniowego układu zamkniętego.

		Rozwiązaniem układu $n$ równań różniczkowych 1-go rzędu jest (w przypadku nie
		zdegenerowanym) właściwa, $n$-wymiarowa krzywa parametryczna $x(t)$ w $n$-wymiarowej
		rozmaitości różniczkowej stanu zwana trajektorią stanu. Parametrem tej krzywej jest czas $t$.
		
		\begin{verbatim}
			Nieliniowy układ dynamiczny opisany równaniem \eqref{model-dynamiki-ukladu-zamknietego}
			nazywamy autonomicznym, jeżeli funkcja $f^*$ nie zależy jawnie od $t$.
		\end{verbatim}

		W zasadzie wszystkie układy fizyczne nie są autonomiczne (gdyż się starzeją, zużywają), ale
		jeżeli zmiany ich właściwości są wolnozmienne w zestawieniu z dynamiką zmian ich stanu, to
		możemy je traktować jako autonomiczne w określonych przedziałach czasowych.
		Nawet obiekt, którego wystarczająco dokładnym modelem dynamiki jest model liniowy w
		przestrzeni stanu, po zastosowaniu nieliniowych reguł sterowania opartych np. o zasadę
		sprzężeń zwrotnych, staje się nieliniowym systemem dynamicznym. A jeżeli reguła
		sterowania zawiera czynnik czasu, to dodatkowo staje się systemem nieautonomicznym.
		\begin{verbatim}
			OBIEKT LINIOWY + REGULATOR LINIOWY = LINIOWY SYSTEM DYNAMICZNY
		\end{verbatim}

		Obiekt sterowania o macierzach ($A$, $B$) modelu w przestrzeni stanu oraz regulator. Daje w rezultacie
		System autonomiczny, o zmodyfikowanej macierzy stanu $[A+KB]$ – o przesuniętych biegunach
		systemu w następstwie odpowiedniego doboru elementów macierzy $K$. Macierz ta zawiera współczynniki
		wzmocnień odpowiednich pętli ujemnych sprzężeń zwrotnych, stabilizujących obiekt
		niestabilny lub poprawiających jego zapasy stabilności.
		\begin{verbatim}
			OBIEKT LINIOWY + REGULATOR NIELINIOWY = NIELINIOWY SYSTEM DYNAMICZNY
			OBIEKT LINIOWY + REGULATOR PROGRAMOWANY = SYSTEM NIEAUTONOMICZNY
		\end{verbatim}
		Trajektorię stanu może stanowić pojedynczy punkt w przestrzeni stanu zwany jest wówczas
		punktem równowagi trwałej.
		\begin{verbatim}
			Def. Stan $x_r$ nazywamy stanem równowagi trwałej, jeżeli z tego że dla pewnego $t$ $x(t) = x_r$
			wynika, że dla dowolnego $\tau > t x(\tau) = x_r$.
		\end{verbatim}
		Tym samym mamy, że $f(x_r) = 0$, czyli że punkt równowagi trwałej jest punktem stacjonarnym
		układu autonomicznego.
		Istnieją jeszcze inne punkty stacjonarne poza punktami równowagi trwałej np. punkty
		równowagi nietrwałej/chwiejnej.
		Model liniowy ma tylko jeden punkt równowagi trwałej: $x_r = 0$ gdy macierz układu $A$ jest
		pełnego rzędu. W przeciwnym wypadku ma on nieskończenie wiele
		punktów równowagi, czyli gdy defekt rzędu jest równy jeden: $rank(A) = n-1$, to miejscem
		geometrycznym punktów równowagi jest prosta w przestrzeni stanu; a gdy $rank(A) = n-2$, jest
		to pewna płaszczyzna w $n$-wymiarowej przestrzeni stanu, czyli punkty równowagi nie są
		wówczas izolowane – nie są zbiorami miary zero. Ogólnie można ten warunek zapisać, że
		\begin{equation}
			x_r \in Ker(A)
		\end{equation}
		W przypadku modeli nieliniowych rzecz przedstawia się odmiennie. Mogą one mieć więcej niż
		jeden izolowany punkt stacjonarny, w tym więcej niż jeden punkt równowagi trwałej,
		Takie punkty zwane są atraktorami i są lokalnymi centrami tzw. obszarów przyciągania
		bądź też basenów atrakcji. Ilość punktów równowagi nie wiąże się z utratą rzędu
		rozmaitości stanu, nawet tej nieliniowej nieliniowej przestrzeni stanu.
		W przypadku utraty pełnej wymiarowości przez rozmaitość stanu, czyli nastąpienia tzw. defektu rzędu, punkty
		stacjonarne tworzą podrozmaitości osobliwe,
		o wymiarowości zależnej od stopnia defektu rzędu rozmaitości stanu. W takim
		przypadku mówimy o ko-wymiarze podrozmaitości osobliwej. Szczególnym przypadkiem są
		np. krzywe zamknięte jako zbiory stacjonarne tzw. orbity np. kołowe, eliptyczne.
		Jeżeli w konkretnym przypadku mamy kilka izolowanych punktów stacjonarnych, to
		każdorazowo dokonuje się równoległego przeniesienia układu współrzędnych stanu
		tak, aby jego początek pokrywał się z wybranym do analizy punktem stacjonarnym.
		Mówimy wówczas o lokalnym, związanym z danym punktem układzie współrzędnych stanu.
		Redefiniujemy wówczas zmienne stanu w następujący sposób:
		\begin{equation}
			\xi(t) = x(t)-x_r \to x(t) = \xi(t) + x_r \to \xi = f^*(\xi+x_r),	
		\end{equation}
	\section{Ruch, trajektoria nominalna, stabilność ruchu względem trajektorii nominalnej}
		Rozważmy następujące pytanie
\end{document}
