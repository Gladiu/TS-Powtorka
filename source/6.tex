\author{Kasiński}
\title{Wykład 6}
\documentclass{article}
\usepackage[margin=1.4cm]{geometry}
\usepackage[utf8]{inputenc}
\usepackage{babel}
\usepackage{polski}
\usepackage{float}
\usepackage{graphicx}
\usepackage{amsmath}
\begin{document}
	\maketitle
	\section{Wprowadzenie}
		Dyskretne w czasie zmiany wartości sygnałów sterujących oraz dyskretne w czasie
		odczytywanie wartości sygnałów mają miejsce w komputerowych systemach bezpośredniego
		sterowania cyfrowego. Mowa tutaj o DDC – od ang. Direct Digital Control

		W przypadku DDC, o chwili modyfikacji wartości sygnału sterującego 
		decyduje zegar czasu rzeczywistego. Zegar ten synchronizujący proces próbkowania
		sygnałów w przypadku użycia przetworników A/C, oraz proces zmian wartości sygnałów
		sterujących w przypadku użycia przetworników C/A.

		Dla uproszczenia pomijamy drugi aspekt digitalizacji, czyli kwantowanie wartości
		sygnałów, zakładając, że rozdzielczość używanych w systemie przetworników A/C i C/A
		jest na tyle wysoka, że efekty związane z szumem kwantowania „w pionie” są pomijalne
		Np. liniowy przetwornik 12-bitowy dzieli zakres zmienności sygnału np. <-10, +10> V
		na 4096 części, co daje amplitudę szumu zaokrąglenia wartosci nie przekraczającą 5mV,
		na ogół znacznie mniejszą niż amplituda szumów generowanych prze inne źródła, np.
		szumy fluktuacji termicznej.

		Zakładamy, że częstotliwość próbkowania w systemie została dobrana zgodnie z
		kryterium Nyquista, na podstawie wyznaczonej częstotliwości granicznej widma,
		ustalonej dla całego zbioru sygnałów w systemie.
		Próbkowanie w układach rzeczywistych odbywa się z nadmiarem – częstotliwość
		próbkowania jest zazwyczaj znaczną wielokrotnością
		(rzędu 20) częstotliwości
		granicznej, tak więc można założyć, że próbkowanie na osi czasu nie prowadzi do utraty
		informacji o procesie, a cyfrowy sposób sterowania nie wprowadza do obiektu istotnych
		efektów szumowych związanych z digitalizacją sygnałów sterujących.

		Pewien problem stanowi zróżnicowana dynamika obiektu sterowania, w której możemy
		wyodrębnić podsystemy wolnozmienne (o dużych wartościach stałych czasowych i
		okresów drgań własnych tworzących je elementów) oraz podsystemy szybkozmienne.
		Możliwe jest wówczas stosowanie odrębnych częstotliwości próbkowania dla tychpodsystemów,
		dostosowanych do każdej z dynamik. Mówimy wówczas, że sterowanie
		przebiega w dwóch skalach czasu, albo w dwóch warstwach.
		
		Należy zadbać o synchronizację algorytmów sterujących w poszczególnych warstwach,
		zwłaszcza wtedy, gdy charakterystyki widmowe podsystemów obu klas znacząco na
		siebie nachodzą. Można dokonać dalej idącej stratyfikacji systemu na warstwy 
		o ile ma to uzasadnienie w postaci jeszcze bardziej zróżnicowanej struktury
		dynamicznej obiektu sterowania.
	\section{Kwestia niestacjonarności obiektów sterowania}
		W dotychczas rozpatrywanych modelach zakładano stałość w czasie parametrów. Są
		jednak sytuacje, gdy urządzenia ulegają widocznemu w czasie zużyciu lub podlegają
		szybkiemu starzeniu. W dotychczas analizowanych modelach czynnik ten nie był
		uwzględniany. Jedynymi zmiennymi w czasie elementami modelu były sygnały.

		Aby móc analizować również obiekty, których właściwości są w istotny sposób zmienne
		w czasie wprowadza się rozszerzenie parametrycznych modeli w przestrzeni stanu
		poprzez uwzględnienie również zmienności parametrów modelu w czasie. Cechą tej
		zmienności jest niska dynamika (procesy starzenia, czy zużycia są stosunkowo wolne w
		porównaniu z dynamiką zmian stanu pod wpływem sygnałów sterujących), co
		odpowiada warunkom rzeczywistym. Ponadto funkcje zużycia/starzenia są stosunkowo
		proste, przeważnie monotoniczne. Dlatego w systemach sterowania procesy te można
		uwzględniać w dodatkowej warstwie super-wolnozmiennej, odpowiadającej za
		aktualizację modeli wykorzystywanych do sterowania.
		
		Równania stanu i wyjścia nieliniowego obiektu niestacjonarnego zapisujemy w postaci
		ogólnej:
		\begin{align*}
			\dot{x}(t) = f(x(t), u(t), t) \\
			y(t) = h(x,u,t)
		\end{align*}
		gdzie argument t oznacza jawną zależność funkcji f, h lub ich parametrów od czasu.
		Oczywiście, niestacjonarność dodatkowo podnosi stopień trudności rozwiązania
		nieliniowych równań stanu.
		\begin{align*}
			\dot{x}(t) = A(t) x(t) + B(t)u(t) \\
			y(t) = C(t) x(t) + D(t) u(t)
		\end{align*}
		gdzie niestacjonarność wbudowno w parametry modelu
		Liniowe równanie układu niestacjonarnego można rozwiązać poprzez całkowanie.
		Rozwiązanie ma wtedy postać:
		\begin{equation}
			x(t) = K(t, 0)x(0) + \int^t_0K(t,\tau)B(\tau)u(\tau)d\tau
		\end{equation}
		Gdzie $K(t,\tau)$ jest nieosobliwą macierzą Cauchiego wymiarach n na n, stanowiącą
		rozwiązanie liniowego, jednorodnego, macierzowego równania różniczkowego o postaci:
		\begin{equation}
			\dot{K}(t, t_0) = A(t)K(t,t_0)
		\end{equation}
		Przy warunku początkowym $K(t_0, t_0) = I$
	\section{Modele dyskretne obiektów z czasem ciągłym}
		Wobec tego, że decyzje sterujące oraz pomiary są
		podejmowane w dyskretnych chwilach czasowych, z odstępem w czasie zwanym
		okresem próbkowania $Tp$, naturalne jest posługiwaniem się do opisu obiektu modelami
		dyskretnymi w czasie, wykorzystywanymi następnie w algorytmach sterowania. W
		szczególności dotyczy to modeli w przestrzeni stanu.

		Istnieją dwa podejścia do dyskretyzacji: aproksymacyjne i dokładne. Podejście
		aproksymacyjne polega na zastąpieniu chwilowej prędkości zmian stanu po lewej stronie
		równania, ilorazem różnicowym przyrostu wektora stanu opisanym jako $\Delta x$
		i skończonego odcinka czasu w którym ten przyrost nastąpił $T_p$. Prowadzi to do
		przybliżonego schematu całkowania Eulera.
		\begin{equation}
			\frac{\Delta x}{T_p}
		\end{equation}
		Taka aproksymacyja skutkuje kumulacją błędu całkowania
		wraz ze wzrostem liczby kroków procedury rekurencyjnej. Jego zaletą jest możliwość
		stosowania go, gdy nie potrafimy uzyskać rozwiązania równania stanu w jawnej postaci.
		Brak możliwośći uzyskania rozwiązania równań stanu w jawnej postaci ma miejsce w
		przeważającej liczbie modeli nieliniowych.
		\begin{verbatim}
			W tym miejscu na wykładzie jest wyprowadzany euler w tył
			Przejdę od razu do wniosków
		\end{verbatim}
		
		\begin{align*}
			x_k = x_{k-1} + T_pf(x_{k-1}, u_{k-1}) \\
			... \\
			x_k \approx [I + AT_p] x_{k-1} + B u_{k-1}
		\end{align*}
		Warto zauważyć, że macierz stanu modelu dyskretnego jest identyczna z pierwszymi dwoma
		wyrazami rozwinięcia w szereg Taylora macierzowej funkcji wykładniczej
		\begin{equation}
			e^{AT_p} \approx I +A^T_p .
		\end{equation}
		A więc zastąpienie pochodnej ilorazem różnicowym ma podobne skutki jak ograniczenie
		rozwinięcia macierzy tranzycji stanu do dwóch pierwszych wyrazów jej rozwinięcia w
		szereg Taylora.
		(Na koniec poprzedniego wykładu te równania są wyprowadzone)

		Przy tej metodzie dyskretyzacji macierz sterowań nie ulega modyfikacji, jest taka sama
		jak w modelu z czasem ciągłym. Natomiast macierz stanu modelu dyskretnego ma
		składnik zależny od okresu próbkowania, co wpływa na rozkład jej wartości własnych,
		tożsamych z biegunami dyskretnej transmitancji macierzowej, o określonym rozkładzie
		na płaszczyźnie zespolonej "z". Niewłaściwy dobór okresu próbkowania może zatem
		doprowadzić do utraty stabilności modelu dyskretnego – transmitancji Laurenta, czyli
		wyjścia jej biegunów poza okrąg jednostkowy, pomimo że model z czasem ciągłym był
		stabilny.
	\section{Dyskretyzacja dokładna}
		Metoda dokładna polega na dyskretyzacji rozwiązania równań stanu. Można ją
		zastosować jedynie wówczas, gdy rozwiązanie ma jawną postać. Jej zaletą jest brak
		kumulacyjnego błędu całkowania przybliżonego, stąd jej nazwa. Można ja zastosować w
		przypadku gdy model stanowy układu z czasem ciągłym ma liniowe równanie stanu.
		Tego warunku nie musi spełniać równanie wyjścia. Rozwiązanie liniowego równania
		stanu ma postać:
		\begin{equation}
			x(t) = e^{At}x(0)+\int^t_0e^{A(t-\tau)}Bu(\tau)d\tau
		\end{equation}
		Przyjmując jako warunek początkowy $x_k$ i całkując w granicach $[kT_p , (k+1)T_p]$
		otrzymujemy równanie rekurencyjne przejścia stanu dla pojedynczego okresu
		próbkowania. Należy też uwzględnić, że w trakcie tego okresu wektor sygnałów sterujących 
		pozostaje stały, równy $u_k$. Sygnał ten zgodnie z mechanizmem bezpośredniego
		sterowania cyfrowego sygnały sterujące mają charakter schodkowy, o szerokości stopnia
		równej $T_p$.
		\begin{align*}
			x_{k+1} = e^{AT_p}x_k+\int^{T_p}_0e^{A(T_p-\tau)}Bu_kd\tau \\
			x_{k+1} = e^{AT_p}x_k+\int^{T_p}_0 e^{A(T_p-\tau)}d\tau Bu_k
		\end{align*}
		Po scałkowaniu otrzymujemy
		\begin{equation}
			x_{k+1} = A^*x_k+B^*U_k
		\end{equation}
		Gdzie
		\begin{itemize}
			\item $A^* = e^{AT_p}$
			\item $B^* = e^{AT_p}\frac{B}{T_p}$
		\end{itemize}
		Macierze $C^*$ oraz $D^*$ nie ulegają zmianie przy dyskretyzcji

		Odtąd nie będziemy stosować gwiazdek w oznaczeniach macierzy modelu dyskretnego
		uzyskanego metodą dokładną, ale będziemy każdorazowo zaznaczać jaką metodą został
		uzyskany konkretny model. Wpływ wartości $T_p$ na stabilność modelu dyskretnego jest
		nadal istotny, jest jednak bardziej złożony niż w przypadku przybliżonego modelu
		dyskretnego. Charakterystyczna jest też zależność odwrotnie proporcjonalna elementów
		macierzy sterowań modelu dyskretnego (macierzy wzmocnień sygnałów sterujących)
		od okresu próbkowania $T_p$, której sens jest następujący: przy tych samych przyrostach
		sygnałów sterujących, efekt sterowania jest tym
		słabszy im dłuższy jest okres próbkowania. Dzieje się tak ponieważ w czasie rzeczywistym
		, sterowanie jest
		korygowane rzadziej z uwagi na rzadsze zamykanie pętli sprzężenia zwrotnego w
		obwodach regulacji.
		\subsection{Dygresja o czymś co było wcześniej udowodnione (?)}
			Analogiczne rozumowanie prowadzące do wyprowadzenia dyskretnego modelu
			dokładnego, można przeprowadzić w odniesieniu do liniowych modeli niestacjonarnych
			z czasem ciągłym. Macierze modelu dyskretnego uzyskanego metodą dokładną (przez
			dyskretyzację jawnego rozwiązania równań stanu) będą w oczywisty sposób związane z
			indeksami próbek, odzwierciedlając w ten sposób niestacjonarność modelu):
			\begin{equation}
				x_{k+1} = Ax_k + Bu_k
			\end{equation}
			Gdzie $A = K[t_{k+1}, t_k]$ natomiast $B = \int_{t_k}^{t_k+T_p}K(t_k+T_p,\tau)B(\tau)d\tau$
\end{document}
